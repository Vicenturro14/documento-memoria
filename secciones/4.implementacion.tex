\chapter{Implementación}
En este capítulo se describirá la implementación realizada durante la memoria.
Esta se separa puede separar en tres partes, la primera es la implementación relacionada
con los estudiantes del ramo Trabajo de Título, también llamado F. La segunda parte
corresponde a la asignación de comisiones examinadoras. La tercera parte corresponde a la
exportación de comisiones examinadoras.

\section{Estudiantes de Trabajo de Título}

\subsection{Módulo de Trabajo de Título}
Para implementar funcionalidades relacionadas con el ramo Trabajo de Título y
separarlas de las funcionalidades del ramo Introducción al Trabajo de Título, se creó el
módulo de Trabajo de Título. Esta separación también evita que la barra de navegación se
llene de pestañas. Para acceder al módulo, se creó un selector en la barra de navegación,
que se ubica en la parte superior de la página, como se puede ver en la figura
\ref{fig:implementacion_selector}. Como por el momento las únicas funcionalidades asociadas
al módulo de Trabajo de Título son ver los estudiantes de Trabajo de Título y la asignación
de comisiones examinadoras, y ambas requieren permisos de coordinación de titulación, el
selector solo es visible para coordinadores, que está manejado por la variable booleana
\verb|show_module_selector|, como se muestra en la línea 11 del código \ref{lst:barra_navegacion}.
Al seleccionar el módulo Introducción al Trabajo de Título, el usuario es redirigido a la
raíz de la aplicación. Mientras que al seleccionar el módulo Trabajo de Título, el usuario
es redirigido a la vista principal de Trabajo de Título, que es el listado de estudiantes de
Trabajo de Título.

Como anteriormente todo el sistema estaba dedicado al ramo Introducción al Trabajo de
Título, el título de la barra de navegación era el nombre del módulo. Por lo tanto, se
creó el módulo Trabajo de Título para poder separarlo del módulo de Introducción al
Trabajo de Título. Además, el título de la barra de navegación se cambió a un título
configurable mediante la variable \verb|titulo| y si esta variable no se especifica, se
muestra el nombre del módulo. Se decidió dejar este caso, ya que el header es el mismo para
todos los módulos y el resto de módulos no requieren un título personalizado.


\begin{figure}[ht]
    \begin{subfigure}[b]{0.47\textwidth}
        \centering
        \includegraphics[width=\linewidth]{imagenes/implementaciones/selector_seccion_cerrado.png}
        \caption{Captura de pantalla del selector cerrado en la barra de navegación en el módulo de Introducción al Trabajo de Título.}
        \label{fig:implementacion_selector_cerrado}
    \end{subfigure}
    \hfill
    \begin{subfigure}[b]{0.47\textwidth}
        \centering
        \includegraphics[width=\linewidth]{imagenes/implementaciones/selector_seccion_abierto.png}
        \caption{Captura de pantalla del selector abierto en la barra de navegación en el módulo de Introducción al Trabajo de Título.}
        \label{fig:implementacion_selector_abierto}
    \end{subfigure}
    \hfill
    \begin{subfigure}[b]{0.47\textwidth}
        \centering
        \includegraphics[width=\linewidth]{imagenes/implementaciones/selector_seccion_trabajo_titulo.png}
        \caption{Captura de pantalla del selector cerrado en la barra de navegación en el módulo de Trabajo de Título.}
        \label{fig:implementacion_selector_trabajo_titulo}
    \end{subfigure}
    \hfill
    \begin{subfigure}[b]{0.47\textwidth}
        \centering
        \includegraphics[width=\linewidth]{imagenes/implementaciones/selector_seccion_trabajo_titulo_abierto.png}
        \caption{Captura de pantalla del selector abierto en la barra de navegación en el módulo de Trabajo de Título.}
        \label{fig:implementacion_selector_trabajo_titulo_abierto}
    \end{subfigure}
    \caption{Capturas de pantalla del selector en la barra de navegación.}
    \label{fig:implementacion_selector}
\end{figure}

\begin{lstlisting}[language=HTML, caption={Código de la parte central de la barra de navegación}, label={lst:barra_navegacion}]
    <div class="header">
      <a href="" class="logo"><span class="dcc">dc<span>c</span></span></a>
      <div class="d-flex flex-column flex-grow-1 align-items-center pt-2">
        <h1 class="modulo mb-0">
          
            {{ titulo }}
          
            {{ modulo.nombre }}
          
        </h1>
        
          <div class="dropdown-center">
            <button class="btn btn-dark dropdown-toggle" type="button" data-bs-toggle="dropdown">
              {{ modulo.nombre }}
            </button>
            <ul class="dropdown-menu">
              
                <li>
                  <a class="dropdown-item  disabled "
                     href=" #  {{ mod.url }} ">
                    {{ mod.nombre }}
                  </a>
                </li>
              
            </ul>
          </div>
        
      </div>
    </div>
\end{lstlisting}

\subsection{Listado de estudiantes de Trabajo de Título}

La asignación de comisiones examinadoras a los estudiantes de Trabajo de Título, requiere
tener un registro de los estudiantes que están cursando Trabajo de Título junto con sus
respectivos temas de memoria. Por lo tanto, dentro del módulo de Trabajo de Título se
creó una vista que lista los estudiantes que están cursando Trabajo de Título, que se
puede ver en la figura \ref{fig:listado_estudiantes_f}.

En esta vista se muestra una tabla que por cada estudiante muestra su nombre, el título
de su memoria, su guía, su coguía si tiene y un botón que redirige a su ficha. Sobre la
tabla al costado derecho se encuentra una barra de búsqueda que permite buscar por nombre
de estudiante, título de memoria, nombre de guía y nombre de coguía. Sobre esta barra, se
encuentra un filtro que permite elegir el semestre académico de los estudiantes que se
mostrarán en la tabla. Por defecto se mostrarán los estudiantes del semestre académico
activo. Por último, al costado derecho de este filtro está el botón de
\textit{Sincronizar}, que permite actualizar la tabla con los estudiantes.

\begin{figure}[ht]
    \centering
    \includegraphics[width=\linewidth]{imagenes/implementaciones/listado_estudiantes_f.png}
    \caption{Captura de pantalla de la vista del listado de estudiantes de Trabajo de Título.}
    \label{fig:listado_estudiantes_f}
\end{figure}

Para la actualización del registro se tomó en cuenta que los estudiantes deben estar
cursando alguna sección de los ramos de Trabajo de Título, CC6909 y CC6919, por lo
tanto se puede obtener los nombres de los estudiantes desde la API de UCampus. Además,
los estudiantes deben haber aprobado Introducción al Trabajo de Título, por lo que se
asume que los temas de los estudiantes ya se encuentran registrados en el sistema de
titulación y que también hay una instancia del modelo Propuesta para cada estudiante
asociada a su tema y con estado aprobado, haciendo posible asociar a los estudiantes que
cursan Trabajo de Título con sus temas.

Dada la asunción anterior, para actualizar el registro de estudiantes de Trabajo de
Título dentro del sistema de titulación se creó un comando de administración
\verb|upd_estudiantes_f|, que es llamado por el botón \textit{Sincronizar}.
Este obtiene desde la API de UCampus todas las secciones de
los ramos de Trabajo de Título y para después obtener los integrantes de cada sección.
Luego, por cada integrante se busca la instancia del modelo Propuesta con estado
aprobado más reciente asociada al estudiante, pues puede haber varias instancias
asociadas al mismo estudiante. Si se encuentra una instancia, se obtiene el tema asociado
y se crea una instancia del modelo MemoriaEnF con el estudiante y su tema. En caso
contrario, se registra en el log que para ese estudiante no se encontró una instancia,
se agrega al estudiante a la lista de estudiantes sin una instancia y se continua con el
siguiente integrante. Al finalizar el proceso se muestra un mensaje indicando cuántos
estudiantes no tuvieron una instancia, como se ve en la figura \ref{fig:estudiante_sin_tema}.
Una vez que termina la ejecución del comando, se recarga la página y se muestra la lista
actualizada de estudiantes de Trabajo de Título.

\begin{figure}[ht]
    \centering
    \includegraphics[width=0.6\linewidth]{imagenes/implementaciones/estudiante_sin_tema.png}
    \caption{Captura de pantalla de la notificación de que un estudiante no tiene un tema registrado.}
    \label{fig:estudiante_sin_tema}
\end{figure}

Como se mencionó en la descripción del comando, puede suceder que haya estudiantes que no
tengan una instancia de Tema y Propuesta con estado Aprobado. Esto sucede por lo que la
asunción puede no ser cierta para estudiantes que hayan solicitado la vía rápida, pues
actualmente ese proceso no se encuentra implementado en el sistema de titulación y se
realiza directamente con la coordinación de titulación. El caso de la vía rápida no
se manejará en esta memoria, ya que al momento del desarrollo de este trabajo de título,
había otro tema de memoría sobre la incorporación de un módulo de vía rápida. Mientras
que tanto, basta con crear una instancia de Tema y Propuesta con estado Aprobado para
cada estudiante de vía rápida mediante el administrador de Django o por consola para que
puedan ser incluidos en la lista de estudiantes de Trabajo de Título.




\begin{itemize}
    \item Listado de estudiantes del F para académicos
\end{itemize}

\section{Asignación de Comisiones}
\begin{itemize}
    \item Listado de Comisiones (adaptación a modelo MemoriaEnF)
    \item Gráfico de Carga
    \item Correcciones: Modal
    \item Correcciones: Evaluadores
\end{itemize}

\subsection{Filtro de comisiones}
Se implementó el filtro de comisiones examinadoras en la interfaz principal del módulo de
asignación de comisiones examinadoras. Este filtro permite configurar si se muestran
todos los trabajos de título, solo los que tienen comisión incompleta o solo los que
tienen comisión completa. En la figura \ref{fig:implementacion_filtro_comisiones} se
puede ver el filtro mostrando solo las memorias con comisión incompleta.

\begin{figure}[ht]
    \centering
    \includegraphics[width=0.8\linewidth]{imagenes/implementaciones/filtro_comisiones.png}
    \caption{Captura de pantalla de la interfaz principal del módulo de asignación de comisiones examinadoras con el filtro de comisiones examinadoras.}
    \label{fig:implementacion_filtro_comisiones}
\end{figure}


\section{Exportación de Comisiones}
\begin{itemize}
    \item CSV
          \begin{itemize}
              \item Ajuste de columnas para SSM
          \end{itemize}
    \item API
          \begin{itemize}
              \item Titulación
                    \begin{itemize}
                        \item Endpoint y comando
                        \item Requiere autenticación con Token
                    \end{itemize}
              \item SSM
                    \begin{itemize}
                        \item Agregar DRF a sistema (Usuario y Token)
                        \item Endpoint y validaciones
                        \item Normalizar nombres al buscar y guardar
                    \end{itemize}
          \end{itemize}
\end{itemize}
