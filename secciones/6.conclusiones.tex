\chapter{Conclusiones}
El desarrollo del módulo de asignación de comisiones examinadoras en el sistema de
titulación del DCC tuvo como objetivo general desarrollar y desplegar una herramienta que
permita asignar comisiones examinadoras a los estudiantes de Trabajo de Título de forma
eficiente y que se comunique con los sistemas existentes relacionados con el proceso de
titulación. Este objetivo fue logrado con éxito, ya que se implementó una extensión del
Sistema de Titulación del DCC que permite a la coordinadora de titulación asignar
comisiones examinadoras a los estudiantes de Trabajo de Título en menor tiempo que el
método anterior de planillas Excel y que es capaz de exportar las comisiones asignadas
al Sistema de Seguimiento de Memorias por dos métodos: la descarga de un archivo CSV y
mediante una consulta POST a la API del Sistema de Seguimiento de Memorias. Además, las
implementaciones realizadas al Sistema de Titulación y al Sistema de Seguimiento de Memorias
fueron desplegadas en el servidor de producción del DCC, dejándolas listas para su uso.

Dentro de los aprendizajes obtenidos, se destaca la importancia de tener reuniones periódicas
con la profesora guía para mantener la constancia en el trabajo, ya que inicialmente no
se tenían y en cuanto se fijó un horario para tener reuniones semanales, el ritmo de trabajo
aumentó. Las reuniones, tanto con la profesora guía como con el equipo de desarrollo del DCC,
requirieron de saber expresar lo que se estaba haciendo y los problemas que surgían, para así
obtener ayuda o sugerencias que permitieran resolver las dificultades que se tuvieron.

Uno de los principales desafíos de la memoria fue trabajar sobre código desarrollado por
otras personas, en particular trabajar sobre el código del piloto de asignación de
comisiones examinadoras, ya que no se tenía una documentación detallada del código y al
ser una versión piloto, funcionaba pero tenía errores que se debían corregir y
optimizar. Gran parte del tiempo dedicado al desarrollo se centró en corregir errores que
surgían de casos borde no considerados en el piloto u optimizar el código para que fuera
más eficiente.

En cuanto a trabajos futuros en la asignación de comisiones examinadoras,
primero se puede implementar la descarga de comisiones examinadoras asignadas en varios
archivos CSV separados por curso. Otro trabajo futuro sería incorporar el procesamiento
de solicitudes de vía rápida a través del Sistema de Titulación. Esto afecta al módulo de
Trabajo de Título, ya que se deben haber ingresado al Sistema de Titulación los temas de
los estudiantes que desean solicitar vía rápida, para así poder identificarlos correctamente
como estudiantes de Trabajo de Título y poder asignarles comisiones examinadoras.

A modo de resumen, el desarrollo del módulo de asignación de comisiones examinadoras en el
Sistema de Titulación del DCC tuvo éxito en cumplir su objetivo general, implementando y
desplegando una solución para apoyar la labor de la coordinadora de titulación en la
asignación de comisiones examinadoras. Este trabajo ayuda a que el Departamento de
Ciencias de la Computación tenga herramientas más adecuadas ante el aumento de
estudiantes que ingresan a la carrera de Ingeniería Civil en Computación, como lo son el
Sistema de Titulación y el Sistema de Seguimiento de Memorias. Además, permite que el
Sistema de Titulación del DCC no solo cubra el ramo de Introducción al Trabajo de Título,
sino que también cubra parte del ramo Trabajo de Título. Esto abre puertas para agregar
más funcionalidades al Sistema de Titulación relacionadas con Trabajo de Título.