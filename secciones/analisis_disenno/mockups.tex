
\section{Diseño de Mockups}\label{sec:mockups}
En esta sección se presentan los mockups de las interfaces principales del módulo de
comisiones examinadoras.

\subsection{Interfaz principal del módulo de comisiones examinadoras}
\label{subsec:mockups:comisiones}
El primer diseño en ser realizado fue el de la interfaz principal del módulo de
asignación de comisiones examinadoras. La interfaz original se puede ver en la figura
\ref{fig:ingsoft_comisiones} y mientras que la propuesta se puede ver en la figura
\ref{fig:mockup_listado_comisiones}.

\begin{figure}[ht]
    \centering
    \includegraphics[width=0.77\linewidth]{imagenes/mockups/listado_comisiones.png}
    \caption{Mockup de la interfaz principal del módulo de comisiones examinadoras.}
    \label{fig:mockup_listado_comisiones}
\end{figure}

Ambas interfaces son similares, ya que la tabla mantiene la misma estructura. Comenzando
por la parte superior, se mantiene el selector de periodo académico y el botón de
\textit{Sincronizar}. El botón \textit{Descargar} es equivalente al botón
\textit{Exportar} de la interfaz original, solo cambia el texto y el ícono. Se agregó un
botón con ícono de gráfico que permite mostrar y esconder un gráfico con la cantidad de
comisiones examinadoras y memorias guíadas por cada profesor en el periodo académico
seleccionado, que se muestra en la figura \ref{fig:mockup_grafico_comisiones}, las columnas
están ordenadas de forma descendiente por la cantidad de carga de cada profesor. Como se
trata de varios profesores, se decidió que el gráfico no tenga etiquetas en el eje X,
pues no serían visibles y saturaría el gráfico. En su lugar, se muestra el nombre del profesor y
la cantidad de comisiones examinadoras y memorias guíadas al pasar el mouse por encima de
una barra del gráfico.

\begin{figure}[ht]
    \centering
    \includegraphics[width=0.8\linewidth]{imagenes/mockups/grafico_comisiones.png}
    \caption{Mockup del gráfico de comisiones examinadoras.}
    \label{fig:mockup_grafico_comisiones}
\end{figure}

También se agregó el botón \textit{Publicar} que permite publicar las comisiones
examinadoras en el sistema de titulación, permitiendo que las comisiones asignadas se
muestren en la ficha del estudiante y estén disponibles para exportar. Al presionar
\textit{Publicar}, aparece un modal con un mensaje de confirmación, indicando si es que
quedan memorias sin comisión asignada y cuántas son, en caso de que haya. Esto se puede
ver en la figura \ref{fig:mockup_confirmacion_subida_comisiones}.

\begin{figure}[ht]
    \centering
    \includegraphics[width=0.7\linewidth]{imagenes/mockups/confirmacion_subida_comisiones.png}
    \caption{Mockup de la confirmación de la publicación de comisiones examinadoras.}
    \label{fig:mockup_confirmacion_subida_comisiones}
\end{figure}

Más abajo, al costado del buscador, se agregó un filtro que permite ajustar si en la
tabla se muestran todas las memorias, solo las memorias con comisión incompleta o solo
las memorias con comisión completa. Este filtro se puede ver en la figura \ref{fig:mockup_filtro_comisiones}.

\begin{figure}[ht]
    \centering
    \includegraphics[width=0.8\linewidth]{imagenes/mockups/filtro_comisiones_zoom.png}
    \caption{Mockup del filtro de la interfaz principal del módulo de comisiones examinadoras.}
    \label{fig:mockup_filtro_comisiones}
\end{figure}

Por último, en la tabla se decidió que en los botones de acción, es decir,
\textit{Agregar} y \textit{Editar}, se mostrarán solo los íconos de + y lápiz,
respectivamente, porque se consideró que sin el texto se mantendría la claridad de la
acción que realizan y se disminuiría la densidad de texto en la tabla.


\subsection{Ficha de un estudiante con comisión asignada}
Después, se diseñó la interfaz de la ficha de un estudiante desde la vista de un
estudiante con la comisión examinadora que le ha sido asignada, que se puede ver en la
figura \ref{fig:mockup_ficha_estudiante}. No se cambió nada de la interfaz original,
solo se agregó la sección de la comisión examinadora. Esta sección se ubica bajo la
sección de Tema en una tabla con el mismo formato visual que las otras tablas de la ficha
del estudiante. Cada fila corresponde a un integrante de la comisión examinadora,
indicando el rol dentro de la comisión, una imagen del integrante, su nombre y correo
electrónico.

\begin{figure}[ht]
    \centering
    \includegraphics[width=0.8\linewidth]{imagenes/mockups/ficha_estudiante.png}
    \caption{Mockup de la ficha de un estudiante desde la vista de un estudiante
        con la comisión examinadora asignada.}
    \label{fig:mockup_ficha_estudiante}
\end{figure}