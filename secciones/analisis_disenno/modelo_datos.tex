\section{Modelo de Datos}\label{sec:modelo_datos}
En esta sección se describirá el modelo de datos del sistema de titulación y cómo
evolucionó durante este trabajo de título. Primero se describirá el modelo de datos en su
estado inicial en la subsección \ref{sec:ta:modelo_datos_inicial}, es decir, como estaba
antes de realizar este trabajo de título. Luego, en la subsección \ref{sec:ta:cambios_iniciales_modelo_datos}
se describirán los cambios que se planificaron inicialmente. Por último, se describirán los cambios
que surgieron durante el desarrollo de esta memoria.

\subsection{Modelo de Datos Inicial}\label{sec:ta:modelo_datos_inicial}
El repositorio del sistema de titulación tiene varios módulos y cada uno tiene sus
propias entidades que pueden relacionarse con las de otros módulos. No todas las
entidades son relevantes para la asignación de comisiones examinadoras, por lo
que solo se describirán las entidades que se relacionan con el módulo de comisiones
examinadoras y el módulo de titulación. Es importante mencionar que el modelo de datos
que se describirá a continuación es el modelo en su estado previo a este trabajo de título.

El módulo principal es el módulo de titulación, que se puede ver en la figura
\ref{fig:modelo_titulacion}. Este módulo contiene las entidades Tema, Solicitud,
Propuesta y Documento. Estas representan el tema de un trabajo de título, una solicitud
de un estudiante a un tema, un trabajo de título de un estudiante en el contexto del
ramo de Introducción al Trabajo de Título y los documentos con los informes que deben
entregar los estudiantes, respectivamente.

\begin{figure}[ht!]
    \centering
    \includegraphics[width=0.6\linewidth]{imagenes/diagramas/titulacion.png}
    \caption{Diagrama del modelo de datos del módulo de titulación.}
    \label{fig:modelo_titulacion}
\end{figure}

El módulo de comisiones examinadoras contiene a las entidades Comision y
AlumnoCursandoMemoria, como muestra la figura \ref{fig:modelo_comisiones}. Comisión
representa una comisión examinadora de un trabajo de título. Contiene una relación con la
entidad Tema, que representa el tema del trabajo de título, y una relación con la entidad
Solicitud, que representa la solicitud del tema. Además, tiene una relación de n a n con la
entidad Evaluador del módulo de evaluación, que representa los integrantes de la comisión
examinadora.

\begin{figure}[ht!]
    \centering
    \includegraphics[width=0.7\linewidth]{imagenes/diagramas/comisiones.png}
    \caption{Diagrama del modelo de datos del módulo de comisiones examinadoras.}
    \label{fig:modelo_comisiones}
\end{figure}

AlumnoCursandoMemoria representa a un estudiante que está cursando Trabajo de título.
Contiene al estudiante a través de una relación con la entidad Persona del módulo kernel
y se asocia a una memoria a través de la solicitud del tema. Además, contiene el periodo
académico en el que el estudiante cursa el ramo Trabajo de título y tiene una relación con
la entidad Comision, que representa la comisión examinadora que se le asigna al estudiante.
Por último, tiene el estado de la defensa, que indica si la defensa ha sido aprobada, reprobada o si
la defensa aún no ha sido realizada. Los posibles estados son \textit{pendiente},
\textit{aprobado} y \textit{reprobado}.


El módulo de evaluación contiene a las entidades Evaluador, que corresponden a personas
que evalúan, como los integrantes de una comisión examinadora. El módulo de departamento
proporciona las entidades Funcionario y Estudiante. Funcionario representa a
funcionarios del departamento y en el caso de titulación, representa a académicos del
departamento que son guías o coguías de trabajos de título. Estudiantes representa a
estudiantes del DCC. Las entidades de ambos módulos se pueden ver en la figura
\ref{fig:modelo_departamento}.

\begin{figure}[ht]
    \centering
    \includegraphics[width=0.89\linewidth]{imagenes/diagramas/departamento_evaluacion.png}
    \caption{Diagrama del modelo de datos de los módulos departamento y evaluación.}
    \label{fig:modelo_departamento}
\end{figure}


El módulo de docencia proporciona los periodos académicos y el
módulo de investigación proporciona Area, que representa áreas de conocimiento dentro de
computación y permite asignar áreas de conocimiento tanto a los académicos como a los
temas de trabajo de título. Por último, el módulo kernel proporciona la entidad
Persona, que representa a cualquier usuario de la plataforma.

\begin{figure}[ht]
    \centering
    \includegraphics[width=\linewidth]{imagenes/diagramas/investigacion.png}
    \caption{Diagrama del modelo de datos de los módulos docencia, investigación y kernel.}
    \label{fig:modelo_docencia_investigacion_kernel}
\end{figure}


\subsection{Cambios Iniciales al Modelo de Datos}\label{sec:ta:cambios_iniciales_modelo_datos}

Tomando en cuenta el modelo de datos actual y las funcionalidades que se desea agregar a
la herramienta, el modelo de datos no sufrirá muchos cambios, ya que la mayoría de las
funcionalidades trabajan con datos que ya se encuentran en el modelo. La única
funcionalidad que requiere un cambio es la de publicar las comisiones examinadoras, puesto
que se necesita diferenciar entre comisiones que han sido publicadas y aquellas que no lo
han sido. Para lograr esto, se agregará la entidad ComisionBorrador que tendrá los mismos
atributos que Comision, pero que se diferenciará por el hecho de que no ha sido
publicada. De esta forma, la publicación de comisiones examinadoras por API se realizará con
los datos de la entidad Comision.

\subsection{Cambios al Modelo de Datos Durante el Desarrollo}\label{sec:ta:cambios_modelo_datos_desarrollo}

A continuación se detallará por cada módulo las modificaciones realizadas al modelo de datos que no fueron
planificadas inicialmente, y la razón por la que se realizaron.

En primer lugar, en el módulo de titulación se agregó el atributo tema a la entidad
Propuesta, que corresponde al tema de la memoria. Este se agregó, puesto que
anteriormente solo se podía acceder al tema de una memoria a través de la entidad
Solicitud. Esto no era práctico y tampoco hace sentido, ya que Propuesta
representa una memoria en el ramo de Introducción al Trabajo de Título y toda memoria
debe tener un tema. Esto se puede ver en la figura \ref{fig:modelo_titulacion_nuevo}.

\begin{figure}[ht]
    \centering
    \includegraphics[width=0.7\linewidth]{imagenes/diagramas/modelo_titulacion_nuevo.png}
    \caption{Diagrama del modelo de datos del módulo de titulación.}
    \label{fig:modelo_titulacion_nuevo}
\end{figure}

Luego, en el módulo de docencia, se agregó la entidad Curso, que representa una sección
de un ramo en un periodo académico específico, lo cual se ve reflejado en la figura
\ref{fig:modelo_docencia_nuevo}. Esta entidad tiene los atributos periodo, número de
sección, código del ramo, nombre del ramo, id del ramo e id del curso. Estos dos últimos
corresponden a identificadores dentro de la API de UCampus. Se agregó esta entidad para
identificar el curso al que pertenece una memoria y así poder exportar las comisiones
examinadoras asignadas al SSM que separa las memorias por curso.

\begin{figure}[ht!]
    \centering
    \includegraphics[width=0.32\linewidth]{imagenes/diagramas/modelo_docencia_nuevo.png}
    \caption{Diagrama del modelo de datos del módulo de docencia.}
    \label{fig:modelo_docencia_nuevo}
\end{figure}

En el módulo de comisiones examinadoras se eliminó el atributo solicitud de la entidad
Comision, puesto que no es necesario que desde una comisión se acceda a la solicitud al
tema de la memoria, ya que la solicitud solo es relevante al comienzo del ramo
Introducción al Trabajo de Título y las comisiones examinadoras aparecen recién en el
ramo Trabajo de Título. También se eliminó la entidad ComisionBorrador, puesto que no
es necesario tener una entidad para comisiones examinadoras que no han sido publicadas,
en su lugar, se agregó el atributo booleano publicada a la entidad Comision, que indica
si la comisión ha sido exportada al SSM. Además, se eliminó la entidad
AlumnoCursandoMemoria, ya que la entidad representaba a un estudiante que cursaba una
memoria en el ramo Trabajo de Título, lo cual no era consistente con el modelo de datos,
pues ya existe la entidad Estudiante y su equivalente en el ramo de Introducción al
Trabajo de Título, es decir Propuesta, representa a la memoria en el ramo de Introducción
al Trabajo de Título y no al estudiante que cursaba la memoria. La entidad fue reemplazada
por la entidad MemoriaEnF, que será descrita a continuación.

Por último, se creó el módulo de Trabajo de Título que contiene la entidad MemoriaEnF.
Esta entidad representa una memoria en el ramo de Trabajo de Título y tiene como atributos
al estudiante que realiza la memoria, la comisión examinadora, el tema de la memoria, el periodo
académico y curso en el que se realiza la memoria, el tipo de memoria, el estado de defensa y una
referencia a la entidad Propuesta, que corresponde a la misma memoria, pero en el ramo de
Introducción al Trabajo de Título. Tanto la entidad MemoriaEnF como la entidad Comision
se pueden ver en la figura \ref{fig:modelo_trabajo_titulo_nuevo}.

\begin{figure}[ht]
    \centering
    \includegraphics[width=0.9\linewidth]{imagenes/diagramas/modelo_trabajotitulo_comision_nuevo.png}
    \caption{Diagrama del modelo de datos de los módulos de comisiones examinadoras y trabajo de título.}
    \label{fig:modelo_trabajo_titulo_nuevo}
\end{figure}
