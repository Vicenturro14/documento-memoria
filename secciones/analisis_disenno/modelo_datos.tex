\section{Modelo de Datos}\label{sec:modelo_datos}
En esta sección se describirá el modelo de datos del Sistema de Titulación y cómo
evolucionó durante este trabajo de título. Primero se describirá el modelo de datos en su
estado inicial en la subsección \ref{sec:ta:modelo_datos_inicial}, es decir, cómo estaba
en producción antes de realizar este trabajo de título. Luego, en la subsección
\ref{sec:ta:cambios_iniciales_modelo_datos}
se describirán los cambios que se planificaron inicialmente. Por último, se describirán los cambios
que surgieron durante el desarrollo de esta memoria.

\subsection{Modelo de Datos Inicial}\label{sec:ta:modelo_datos_inicial}
El repositorio del Sistema de Titulación tiene varios módulos y cada uno tiene sus
propias entidades que pueden relacionarse con las de otros módulos. No todas las
entidades son relevantes para la asignación de comisiones examinadoras, por lo
que solo se describirán las entidades que se relacionan con el módulo de Comisiones
Examinadoras y el módulo de Titulación. Es importante mencionar que el modelo de datos
que se describirá a continuación es el modelo en su estado previo a este trabajo de título.

El módulo principal es el módulo de Titulación, que se puede ver en la figura
\ref{fig:modelo_titulacion}. Este módulo contiene las entidades Tema, Solicitud,
Propuesta y Documento. Tema representa el tema de un trabajo de título y Solicitud
corresponde a la solicitud de un estudiante a un tema. Por su parte, Propuesta representa
el trabajo de título de un estudiante en el contexto del ramo de Introducción al Trabajo
de Título, mientras que Documento corresponde a los documentos con los informes que deben entregar los
estudiantes durante el proceso de titulación.


\begin{figure}[ht!]
  \centering
  \includegraphics[width=0.6\linewidth]{imagenes/diagramas/titulacion.png}
  \caption{Diagrama del modelo de datos del módulo de Titulación.}
  \label{fig:modelo_titulacion}
\end{figure}

El módulo de Departamento proporciona las entidades Funcionario y Estudiante. Funcionario
representa a funcionarios del departamento, que en el caso del proceso de titulación,
corresponden a académicos que proponen temas para trabajos de título y son guías o
coguías de estos. Estudiantes representa a estudiantes del DCC, que en el contexto del
proceso de titulación, son quienes solicitan un tema de trabajo de título y lo llevan a
cabo. Las entidades de este módulo se pueden ver en la figura
\ref{fig:modelo_departamento}.

\begin{figure}[ht!]
  \centering
  \includegraphics[width=0.6\linewidth]{imagenes/diagramas/departamento.png}
  \caption{Diagrama del modelo de datos del módulo Departamento.}
  \label{fig:modelo_departamento}
\end{figure}


El módulo de Docencia proporciona los periodos académicos y el
módulo de Investigación proporciona Area, que representa áreas de conocimiento dentro de
computación y permite asignar áreas de conocimiento tanto a los académicos como a los
temas de trabajo de título. Por último, el módulo Kernel proporciona la entidad
Persona, que representa a cualquier usuario de la plataforma. La figura
\ref{fig:modelo_docencia_investigacion_kernel} muestra los modelos de datos recién descritos.

\begin{figure}[ht]
  \centering
  \includegraphics[width=\linewidth]{imagenes/diagramas/investigacion.png}
  \caption{Diagrama del modelo de datos de los módulos Docencia, Investigación y Kernel.}
  \label{fig:modelo_docencia_investigacion_kernel}
\end{figure}


\subsection{Cambios Iniciales al Modelo de Datos}\label{sec:ta:cambios_iniciales_modelo_datos}
En esta subsección se describirán los cambios que se tuvieron en cuenta inicialmente para
agregar el proceso de asignación de comisiones examinadoras al Sistema de Titulación.
Se usaron algunas entidades del modelo de datos del piloto implementado por el equipo de
Ingeniería de Software II mencionado en la sección \ref{sec:sa:asignacion}. Específicamente,
las entidades del módulo de Evaluación y el módulo de Comisiones Examinadoras.
En un comienzo se planeaba usar estas entidades tal cual como se encontraban en el piloto
y se describen a continuación, sin embargo, durante el desarrollo de esta memoria fueron
adaptadas, como se describe en la sección \ref{sec:ta:cambios_modelo_datos_desarrollo}.

El módulo de Evaluación contiene a la entidad Evaluador y
Jerarquía, como se muestra en la figura \ref{fig:modelo_evaluacion}. Evaluador representa
a una persona que tiene la tarea de evaluar, que en el contexto de Trabajo de Título,
corresponde a los integrantes de una comisión examinadora.
Esta entidad tiene como atributos datos personales como el primer nombre, el apellido
paterno, y el email del evaluador. Además, tiene un atributo
que indica si está habilitado para evaluar o no, una referencia a
la entidad Persona que representa al evaluador, una referencia a la entidad Jerarquía que
representa la jerarquía de un evaluador y por último una relación de n a n con la entidad
Area, que representa las áreas de conocimiento de cada evaluador. Por su parte, la entidad Jerarquía representa
la jerarquía de un evaluador y tiene un nombre y una descripción. Actualmente las jerarquías pueden ser
AJC (Académico Jornada Completa), AJP (Académico Jornada Parcial) o PEX (Profesor Experto Externo).

\begin{figure}[ht!]
  \centering
  \includegraphics[width=0.6\linewidth]{imagenes/diagramas/evaluacion.png}
  \caption{Diagrama del modelo de datos del módulo de Evaluación.}
  \label{fig:modelo_evaluacion}
\end{figure}

El módulo de Comisiones Examinadoras, contiene a las entidades
Comision y AlumnoCursandoMemoria, como muestra la figura \ref{fig:modelo_comisiones}.
Comisión representa una comisión examinadora de un trabajo de título. Contiene una relación
con la entidad Solicitud, que representa la solicitud del tema. Además, tiene una relación de n a n con la
entidad Evaluador del módulo de Evaluación, que representa los integrantes de la comisión
examinadora.

\begin{figure}[ht!]
  \centering
  \includegraphics[width=0.8\linewidth]{imagenes/diagramas/comisiones.png}
  \caption{Diagrama del modelo de datos del módulo de Comisiones Examinadoras.}
  \label{fig:modelo_comisiones}
\end{figure}

AlumnoCursandoMemoria representa a un estudiante que está cursando Trabajo de título.
Contiene al estudiante a través de una relación con la entidad Persona del módulo Kernel
y se asocia a una memoria a través de la solicitud del tema. Además, contiene el periodo
académico en el que el estudiante cursa el ramo Trabajo de título y tiene una relación con
la entidad Comision, que representa la comisión examinadora que se le asigna al estudiante.
Por último, tiene el estado de la defensa, que indica si la defensa ha sido aprobada, reprobada o si
la defensa aún no ha sido realizada. Los posibles estados son \textit{pendiente},
\textit{aprobado} y \textit{reprobado}.

Por último, se agregó la entidad ComisionBorrador, para representar a las comisiones examinadoras
que han sido creadas por el coordinador de titulación, pero que aún no han sido publicadas y no
deben ser vistas por los académicos o estudiantes. Por lo tanto, esta entidad tiene los mismos
atributos que la entidad Comision.

\subsection{Cambios al Modelo de Datos Durante el Desarrollo}\label{sec:ta:cambios_modelo_datos_desarrollo}

A continuación se detallará por cada módulo las modificaciones realizadas al modelo de
datos que no fueron planificadas inicialmente, y la razón por la que se realizaron.

En primer lugar, en el módulo de Titulación a la entidad Propuesta se agregó el atributo
tema, que corresponde al tema de la memoria. Este se agregó, puesto que anteriormente
solo se podía acceder al tema de una memoria a través de la entidad Solicitud. Esto no
era práctico y tampoco hace sentido, ya que Propuesta representa una memoria en el ramo
de Introducción al Trabajo de Título y toda memoria debe tener un tema. Esto se puede ver
en la figura \ref{fig:modelo_titulacion_nuevo}.

\begin{figure}[ht]
  \centering
  \includegraphics[width=0.7\linewidth]{imagenes/diagramas/modelo_titulacion_nuevo.png}
  \caption{Diagrama del modelo de datos del módulo de Titulación.}
  \label{fig:modelo_titulacion_nuevo}
\end{figure}

Luego, en el módulo de Docencia, se agregó la entidad Curso, que representa una sección
de un ramo en un periodo académico específico, lo cual se ve reflejado en la figura
\ref{fig:modelo_docencia_nuevo}. Esta entidad tiene como atributos el periodo académico,
el número de sección, el código del ramo, el nombre del ramo, el id del ramo y el id del
curso. Estos dos últimos son atributos que vienen desde UCampus. Se agregó esta entidad
para identificar el curso de Trabajo de Título al que pertenece una memoria, ya que el SSM
requiere de esta información.

\begin{figure}[ht!]
  \centering
  \includegraphics[width=0.32\linewidth]{imagenes/diagramas/modelo_docencia_nuevo.png}
  \caption{Diagrama del modelo de datos del módulo de Docencia.}
  \label{fig:modelo_docencia_nuevo}
\end{figure}

En el módulo de Comisiones Examinadoras se eliminó el atributo solicitud de la entidad
Comision, puesto que no estaba siendo utilizado. También se eliminó la entidad
ComisionBorrador, puesto que no es necesario tener una entidad para comisiones examinadoras
que no han sido publicadas, en su lugar, se agregó el atributo booleano \verb|publicada|
en la entidad Comision, que indica si la comisión ha sido publicada al SSM. Además, se eliminó
la entidad AlumnoCursandoMemoria, ya que no era consistente con el resto del modelo de datos,
porque esta representaba a un estudiante que cursaba una memoria en el ramo Trabajo de Título.
Sin embargo, ya existe la entidad Estudiante para representar a un estudiante. Además, su
equivalente en el ramo de Introducción al Trabajo de Título es la entidad Propuesta, que
representa a una memoria en el ramo de Introducción al Trabajo de Título y no al estudiante que
cursaba la memoria. Es por esto que AlumnoCursandoMemoria fue reemplazada por la entidad
MemoriaEnF, que será descrita a continuación.

Por último, se creó el módulo de Trabajo de Título que contiene la entidad MemoriaEnF.
Esta entidad representa una memoria en el ramo de Trabajo de Título y tiene como atributos
al estudiante que realiza la memoria, la comisión examinadora, el tema de la memoria, el periodo
académico y curso en el que se realiza la memoria, el tipo de memoria y una referencia a la
entidad Propuesta, que corresponde a la misma memoria, pero en el ramo de Introducción al
Trabajo de Título. Tanto la entidad MemoriaEnF como la entidad Comision
se pueden ver en la figura \ref{fig:modelo_trabajo_titulo_nuevo}.

\begin{figure}[ht]
  \centering
  \includegraphics[width=0.9\linewidth]{imagenes/diagramas/modelo_trabajotitulo_comision_nuevo.png}
  \caption{Diagrama del modelo de datos de los módulos de Comisiones Examinadoras y Trabajo de Título.}
  \label{fig:modelo_trabajo_titulo_nuevo}
\end{figure}
\newpage