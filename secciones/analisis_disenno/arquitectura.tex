\section{Arquitectura}\label{sec:arquitectura}
A continuación se describirá la arquitectura del sistema de titulación. Primero se
describirá la arquitectura interna del sistema, es decir, los módulos que componen el
sistema. Luego, se describirá la arquitectura externa del sistema, es decir, cómo se
relaciona con otros sistemas.

El sistema de titulación es un sistema web hecho con el framework Django y utiliza una
base de datos PostgreSQL. Internamente se encuentra formado por varios módulos como se
puede ver en la figura \ref{fig:arquitectura_interna}, cada uno sigue la arquitectura
Modelo-Vista-Controlador (MVC), puesto que el framework Django ya implementa esta
arquitectura. Los módulos son los siguientes:
\begin{multicols}{3}
    \begin{itemize}
        \item Core
        \item Servicios
        \item Comisiones
        \item Titulación
        \item Trabajo de Título
        \item Docencia
        \item Evaluación
        \item Departamento
        \item Investigación
    \end{itemize}
\end{multicols}

El módulo Core actúa como base transversal, conteniendo la configuración y
funcionalidades utilizadas por todo el sistema. Por su parte, el módulo de servicios
gestiona las notificaciones, mientras que el de docencia se ocupa de los periodos
académicos y los cursos. Asimismo, el módulo de investigación administra las áreas de
investigación, y el de departamento es responsable de la gestión de estudiantes,
funcionarios y sus respectivos roles. De igual manera, el módulo de evaluación se encarga
de gestionar a los evaluadores de las memorias en ambos ramos de titulación. El módulo de
titulación abarca la lógica del ramo Introducción al Trabajo de Título, incluyendo temas,
solicitudes y memorias; a su vez, el módulo de trabajo de título se enfoca en el ramo
homónimo. Finalmente, el módulo de comisiones examinadoras maneja lo referente a la
asignación y exportación de dichas comisiones.

\begin{figure}[ht]
    \centering
    \includegraphics[width=0.6\linewidth]{imagenes/arquitectura/arquitectura_interna_titulacion.png}
    \caption{Arquitectura interna del sistema de titulación.}
    \label{fig:arquitectura_interna}
\end{figure}

El sistema de titulación interactúa con otros sistemas, ya sea consumiendo sus APIs o
exportando datos, como se puede ver en la figura \ref{fig:arquitectura_externa}. Específicamente,
el sistema de titulación se relaciona con el Sistema de Seguimiento de Memorias (SSM), la
API de UCampus llamada Mufasa y el Portal DCC.

\begin{figure}[ht]
    \centering
    \includegraphics[width=0.5\linewidth]{imagenes/arquitectura/arquitectura_macro_titulacion.png}
    \caption{Arquitectura externa del sistema de titulación.}
    \label{fig:arquitectura_externa}
\end{figure}

El sistema depende del Portal DCC para el inicio de sesión de los usuarios y la obtención
de los datos de estos, como roles y permisos. Por otro lado, el sistema utiliza la API de
UCampus para obtener los periodos académicos, los ramos relacionados con el proceso de
titulación, sus respectivas secciones y estudiantes inscritos en estas secciones.
Finalmente, el sistema exporta datos sobre las memorias y las comisiones examinadoras
asignadas al Sistema de Seguimiento de Memorias (SSM), mediante una petición POST
autenticada con un token generado por el SSM.