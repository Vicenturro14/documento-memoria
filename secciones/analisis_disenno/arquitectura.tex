\section{Arquitectura}\label{sec:arquitectura}
A continuación se describirá la arquitectura del Sistema de Titulación. Primero se
describirá la arquitectura interna del sistema, es decir, los módulos que componen el
sistema. Luego, se describirá la arquitectura externa del sistema, es decir, cómo se
relaciona con otros sistemas.

El Sistema de Titulación es un sistema web hecho con el framework Django y utiliza una
base de datos PostgreSQL. Internamente se encuentra formado por varios módulos como se
puede ver en la figura \ref{fig:arquitectura_interna}, cada uno sigue la arquitectura
Modelo-Vista-Controlador (MVC), puesto que el framework Django ya implementa esta
arquitectura. Los módulos son los siguientes:
\begin{multicols}{3}
  \begin{itemize}
    \item Core
    \item Servicios
    \item Comisiones
    \item Titulación
    \item Trabajo de Título
    \item Docencia
    \item Evaluación
    \item Departamento
    \item Investigación
  \end{itemize}
\end{multicols}

El módulo Core actúa como base transversal, conteniendo la configuración y
funcionalidades utilizadas por todo el sistema. Por su parte, el módulo de Servicios
gestiona las notificaciones, mientras que el de Docencia se ocupa de los periodos
académicos y los cursos. Asimismo, el módulo de Investigación administra las áreas de
investigación, y el de Departamento es responsable de la gestión de estudiantes,
funcionarios y sus respectivos roles. De igual manera, el módulo de Evaluación se encarga
de gestionar a los evaluadores de las memorias en ambos ramos de titulación. El módulo de
Titulación abarca la lógica del ramo Introducción al Trabajo de Título, incluyendo temas,
solicitudes y memorias; a su vez, el módulo de Trabajo de Título se enfoca en el ramo
homónimo. Finalmente, el módulo de Comisiones Examinadoras maneja lo referente a la
asignación y exportación de dichas comisiones.

\begin{figure}[ht]
  \centering
  \includegraphics[width=0.6\linewidth]{imagenes/arquitectura/arquitectura_interna_titulacion.png}
  \caption{Arquitectura interna del Sistema de Titulación.}
  \label{fig:arquitectura_interna}
\end{figure}

El Sistema de Titulación interactúa con otros sistemas, ya sea consumiendo sus APIs o
exportando datos. Específicamente, el Sistema de Titulación se relaciona con el Sistema
de Seguimiento de Memorias (SSM), la API de UCampus llamada Mufasa y el Portal DCC.

El Portal DCC es una plataforma que permite acceder a varias aplicaciones del DCC, como el
Sistema de Titulación y el Sistema de Seguimiento de Memorias (SSM). Además, provee un
servicio de inicio de sesión único para los sistemas del DCC, es decir, un Single Sign-On (SSO),
permitiendo a los usuarios iniciar sesión con su cuenta Upasaporte de UCampus \cite{SSO_DCC}, y
entregando datos del usuario al sistema una vez que este se ha autenticado. En la figura
\ref{fig:sso_dcc} se muestra el flujo de autenticación al utilizar el SSO del Portal DCC.
En particular, el Sistema de Titulación depende del Portal DCC para el inicio de sesión de los
usuarios y la obtención de sus datos, como nombre, rut, roles y permisos dentro del DCC.

\begin{figure}[ht]
  \centering
  \includegraphics[width=0.7\linewidth]{imagenes/arquitectura/sso_dcc.png}
  \caption{Flujo de autenticación con el SSO del Portal DCC obtenido del repositorio del proyecto Django Utils \cite{SSO_DCC}.}
  \label{fig:sso_dcc}
\end{figure}

Por otro lado, el sistema utiliza la API de UCampus para obtener los periodos académicos,
los ramos relacionados con el proceso de titulación, sus respectivas secciones y estudiantes
inscritos en estas secciones. Finalmente, el sistema exporta datos sobre las memorias y las
comisiones examinadoras asignadas al Sistema de Seguimiento de Memorias (SSM), mediante una
petición POST autenticada con un token generado por el SSM. Un resumen de las interacciones
entre el Sistema de Titulación y los otros sistemas puede verse en la figura
\ref{fig:arquitectura_externa}.

\begin{figure}[ht!]
  \centering
  \includegraphics[width=0.5\linewidth]{imagenes/arquitectura/arquitectura_macro_titulacion.png}
  \caption{Arquitectura externa del Sistema de Titulación.}
  \label{fig:arquitectura_externa}
\end{figure}
\newpage