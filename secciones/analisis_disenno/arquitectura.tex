\section{Arquitectura}\label{sec:arquitectura}
A continuación se describirá la arquitectura del sistema de titulación. Primero se
describirá la arquitectura interna del sistema, es decir, los módulos que componen el
sistema. Luego, se describirá la arquitectura externa del sistema, es decir, cómo se
relaciona con otros sistemas.

El sistema de titulación es un sistema web hecho con el framework Django y utiliza una
base de datos PostgreSQL. Internamente se encuentra formado por varios módulos como se
puede ver en la figura \ref{fig:arquitectura_interna}, cada uno sigue la arquitectura
Modelo-Vista-Controlador (MVC), puesto que el framework Django ya implementa esta
arquitectura. Los módulos son los siguientes:
\begin{multicols}{2}
    \begin{itemize}
        \item Core
        \item Módulo de servicios
        \item Módulo de comisiones examinadoras
        \item Módulo de titulación
        \item Módulo de trabajo de título
        \item Módulo de docencia
        \item Módulo de evaluación
        \item Módulo de departamento
        \item Módulo de investigación
    \end{itemize}
\end{multicols}

El módulo Core contiene configuración y funcionalidades utilizada por todo el sistema. El
módulo de servicios se encarga de las notificaciones. El módulo de docencia se encarga de
gestionar los periodos académicos y los cursos. El módulo de investigación se encarga de
las áreas de investigación. El módulo de departamento se encarga de gestionar los
estudiantes, funcionarios y los roles de estos. El módulo de titulación contiene la
lógica relacionada con el ramo Introducción al Trabajo de Título, como los temas, las
solicitudes de temas y las memorias durante ese ramo. El módulo de trabajo de título
contiene la lógica relacionada con el ramo Trabajo de Título. Por último, el módulo de
comisiones examinadoras contiene la lógica relacionada con las asignación y exportación
de comisiones examinadoras.

\begin{figure}[ht]
    \centering
    \includegraphics[width=0.6\linewidth]{imagenes/arquitectura/arquitectura_interna_titulacion.png}
    \caption{Arquitectura interna del sistema de titulación.}
    \label{fig:arquitectura_interna}
\end{figure}

El sistema de titulación se relaciona con otros sistemas, ya sea consumiendo sus APIs o
exportando datos, como se puede ver en la figura \ref{fig:arquitectura_externa}. Específicamente,
el sistema de titulación se relaciona con el Sistema de Seguimiento de Memorias (SSM), la
API de UCampus llamada Mufasa y el Portal DCC.

\begin{figure}[ht]
    \centering
    \includegraphics[width=0.5\linewidth]{imagenes/arquitectura/arquitectura_macro_titulacion.png}
    \caption{Arquitectura externa del sistema de titulación.}
    \label{fig:arquitectura_externa}
\end{figure}

El sistema depende del Portal DCC para el inicio de sesión de los usuarios y la obtención de
los datos de estos, como roles y permisos. Por otro lado, el sistema utiliza la API de
UCampus para obtener datos de los periodos académicos, los ramos relacionados con el
proceso de titulación, sus respectivas secciones y los estudiantes inscritos en estas
secciones. Finalmente, el sistema exporta datos sobre las memorias y las comisiones
examinadoras asignadas al Sistema de Seguimiento de Memorias (SSM). Esta exportación
se realiza a través de la API de SSM.
