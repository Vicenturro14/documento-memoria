\chapter{Análisis y Diseño}

\section{Modelo de Datos}

\subsection{Modelo de datos actual}
El repositorio del sistema de titulación tiene varios módulos y cada uno tiene sus 
propias entidades que pueden relacionarse con las de otros módulos. No todas las
entidades son relevantes el módulo de asignación de comisiones examinadoras, por lo que
solo se describirán las entidades que se relacionan con el módulo de asignación de
comisiones examinadoras y el módulo de titulación. Es importante mencionar que el modelo
de datos que se describirá a continuación es el modelo en su estado previo a este trabajo
de título.

El módulo principal es el módulo de titulación, que se puede ver en la figura 
\ref{fig:modelo_titulacion}. Este módulo contiene las entidades Tema, Solicitud, 
Propuesta y Documento. Estas representan el tema de un trabajo de título, una solicitud 
de un estudiante a un tema, un trabajo de título de un estudiante en el contexto del 
ramo de Introducción al Trabajo de Título y los documentos con los informes que deben 
entregar los estudiantes, respectivamente.

\begin{figure}[ht]
    \centering
    \includegraphics[width=\linewidth]{imagenes/diagramas/titulacion.png}
    \caption{Diagrama del modelo de datos del módulo de titulación.}
    \label{fig:modelo_titulacion}
\end{figure}

El módulo de comisiones examinadoras contiene a las entidades Comision y 
AlumnoCursandoMemoria, como muestra la figura \ref{fig:modelo_comisiones}. Comisión 
representa una comisión examinadora de un trabajo de título. Contiene una relación con la 
entidad Solicitud, que representa el tema del trabajo de título, y una relación con la entidad 
Solicitud, que representa la solicitud del tema. Además, tiene una relación de n a n con la 
entidad Evaluador del módulo de evaluación, que representa los integrantes de la comisión
examinadora.

AlumnoCursandoMemoria representa a un estudiante que está cursando Trabajo de título. 
Contiene al estudiante a través de una relación con la entidad Persona del módulo kernel 
y se asocia a una memoria a través de la solicitud del tema. Además, contiene el periodo 
académico en el que el estudiante cursa el ramo Trabajo de título y tiene una relación con 
la entidad Comision, que representa la comisión examinadora que se le asigna al estudiante.
Por último, tiene el estado de la defensa, que indica si la defensa ha sido aprobada, reprobada o si
la defensa aún no ha sido realizada. Los posibles estados son \textit{pendiente}, 
\textit{aprobado} y \textit{reprobado}.

\begin{figure}[ht]
    \centering
    \includegraphics[width=\linewidth]{imagenes/diagramas/comisiones.png}
    \caption{Diagrama del modelo de datos del módulo de asignación de comisiones examinadoras.}
    \label{fig:modelo_comisiones}
\end{figure}

El módulo de evaluación contiene a las entidades Evaluador, que corresponden a personas
que evaluan, como los integrantes de una comisión examinadora. El módulo de departamento
proporciona las entidades Funcionario y Estudiante. Funcionario representa a 
funcionarios del departamento y en el caso de titulación, representa a académicos del 
departamento que son guías o coguías de trabajos de título. Estudiantes representa a 
estudiantes del DCC.

\begin{figure}[ht]
    \centering
    \includegraphics[width=0.8\linewidth]{imagenes/diagramas/departamento_evaluacion.png}
    \caption{Entidades de los módulos departamento y evaluación.}
    \label{fig:modelo_departamento}
\end{figure}


El módulo de docencia proporciona los periodos académicos y el 
módulo de investigación proporciona Area, que representa áreas de conocimiento dentro de 
computación y permite asignar áreas de conocimiento tanto a los académicos como a los 
temas de trabajo de título. Por último, el módulo kernel proporciona la entidad 
Persona, que representa a cualquier usuario de la plataforma.

\begin{figure}[ht]
    \centering
    \includegraphics[width=0.8\linewidth]{imagenes/diagramas/investigacion.png}
    \caption{Entidades de los módulos docencia, investigación y kernel.}
    \label{fig:modelo_docencia_investigacion_kernel}
\end{figure}


\subsection{Cambios en el modelo de datos}

Tomando en cuenta el modelo de datos actual y las funcionalidades que se desea agregar a 
la herramienta, el modelo de datos no sufrirá muchos cambios, ya que la mayoría de las
funcionalidades trabajan con datos que ya se encuentran en el modelo. La única 
funcionalidad que requiere un cambio es la de publicar las comisiones examinadoras, puesto
que se necesita diferenciar entre comisiones que han sido publicadas y aquellas que no lo
han sido. Para lograr esto, se agregará la entidad ComisionBorrador que tendrá los mismos
atributos que Comision, pero que se diferenciará por el hecho de que no ha sido
publicada. De esta forma, la publicación de comisiones examinadoras por API se realizará con
los datos de la entidad Comision.


\section{Diseño de Mockups}\label{sec:ta:mockups}
Es importante recalcar que la herramienta es una extensión del trabajo realizado por un 
equipo de Ingeniería de Software II y del sistema de titulación, por lo tanto, en la 
mayoría de los mockups se utilizan las interfaces ya existentes como base.

\subsection{Interfaz principal del módulo de comisiones examinadoras}
\label{subsec:ta:mockups:comisiones}
El primer diseño en ser realizado fue el de la interfaz principal del módulo de 
asignación de comisiones examinadoras. La interfaz original se puede ver en la figura
\ref{fig:ingsoft_comisiones} y mientras que la propuesta se puede ver en la figura
\ref{fig:mockup_listado_comisiones}.

\begin{figure}[ht]
    \centering
    \includegraphics[width=0.77\linewidth]{imagenes/mockups/listado_comisiones.png}
    \caption{Mockup de la interfaz principal del módulo de comisiones examinadoras.}
    \label{fig:mockup_listado_comisiones}
\end{figure}

Ambas interfaces son similares, ya que la tabla mantiene la misma estructura. Comenzando
por la parte superior, se mantiene el selector de periodo académico y el botón de
\textit{Sincronizar}. El botón \textit{Descargar} es equivalente al botón 
\textit{Exportar} de la interfaz original, solo cambia el texto y el ícono. Se agregó un 
botón con ícono de gráfico que permite mostrar y esconder un gráfico con la cantidad de
comisiones examinadoras y memorias guíadas por cada profesor en el periodo académico 
seleccionado, que se muestra en la figura \ref{fig:mockup_grafico_comisiones}, las columnas
están ordenadas de forma descendiente por la cantidad de carga de cada profesor. Como se
trata de varios profesores, se decidió que el gráfico no tenga etiquetas en el eje X, 
pues no serían visibles y saturaría el gráfico. En su lugar, se muestra el nombre del profesor y 
la cantidad de comisiones examinadoras y memorias guíadas al pasar el mouse por encima de
una barra del gráfico.

\begin{figure}[ht]
    \centering
    \includegraphics[width=0.8\linewidth]{imagenes/mockups/grafico_comisiones.png}
    \caption{Mockup del gráfico de comisiones examinadoras.}
    \label{fig:mockup_grafico_comisiones}
\end{figure}

También se agregó el botón \textit{Publicar} que permite publicar las comisiones 
examinadoras en el sistema de titulación, permitiendo que las comisiones asignadas se
muestren en la ficha del estudiante y estén disponibles para exportar. Al presionar 
\textit{Publicar}, aparece un modal con un mensaje de confirmación, indicando si es que
quedan memorias sin comisión asignada y cuántas son, en caso que que hayan. Esto se puede
ver en la figura \ref{fig:mockup_confirmacion_subida_comisiones}.

\begin{figure}[ht]
    \centering
    \includegraphics[width=0.7\linewidth]{imagenes/mockups/confirmacion_subida_comisiones.png}
    \caption{Mockup de la confirmación de la publicación de comisiones examinadoras.}
    \label{fig:mockup_confirmacion_subida_comisiones}
\end{figure}

Más abajo, al costado del buscador, se agregó un filtro que permite ajustar si en la 
tabla se muestran todas las memorias, solo las memorias con comisión incompleta o solo
las memorias con comisión completa. Este filtro se puede ver en la figura \ref{fig:mockup_filtro_comisiones}.

\begin{figure}[ht]
    \centering
    \includegraphics[width=0.8\linewidth]{imagenes/mockups/filtro_comisiones_zoom.png}
    \caption{Mockup del filtro de la interfaz principal del módulo de comisiones examinadoras.}
    \label{fig:mockup_filtro_comisiones}
\end{figure}

Por último, en la tabla se decidió que en los botones de acción, es decir, 
\textit{Agregar} y \textit{Editar}, se mostrarán solo los íconos de + y lápiz,
respectivamente, porque se consideró que sin el texto se mantendría la claridad de la
acción que realizan y se disminuiría la densidad de texto en la tabla.


\subsection{Ficha de un estudiante con comisión asignada}
Después, se diseñó la interfaz de la ficha de un estudiante desde la vista de un 
estudiante con la comisión examinadora que le ha sido asignada, que se puede ver en la 
figura \ref{fig:mockup_ficha_estudiante}. No se cambió nada de la interfaz original,
solo se agregó la sección de la comisión examinadora. Esta sección se ubica bajo la 
sección de Tema en una tabla con el mismo formato visual que las otras tablas de la ficha
del estudiante. Cada fila corresponde a un integrante de la comisión examinadora, 
indicando el rol dentro de la comisión, una imagen del integrante, su nombre y correo 
electrónico.

\begin{figure}[ht]
    \centering
    \includegraphics[width=0.8\linewidth]{imagenes/mockups/ficha_estudiante.png}
    \caption{Mockup de la ficha de un estudiante desde la vista de un estudiante
    con la comisión examinadora asignada.}
    \label{fig:mockup_ficha_estudiante}
\end{figure}