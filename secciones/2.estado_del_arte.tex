\chapter{Situación Actual}

Actualmente existen varios proyectos y sistemas relacionados con el proceso de titulación
en el DCC. Dentro de estos se encuentran el sistema de titulación del DCC, el Sistema de
Monitoreo de Memorias \cite{SistemaMonitoreoMemorias}, un sistema de recomendación de
comisiones \cite{SistemaRecomendacion} y un sistema de asignación de comisiones
desarrollado por un grupo del curso CC5401 Ingeniería de Software II. A continuación se
describen las características y las limitaciones de estos sistemas, relacionadas con la
asignación de comisiones examinadoras.


\section{Sistema de Titulación DCC}\label{sec:sa:titulacion}

El sistema de titulación del DCC es un sistema web en producción que ofrece distintas
funcionalidades dependiendo de si se es estudiante, profesor o coordinador de titulación.
Como estudiante, se pueden ver un listado de temas para trabajos de título, solicitar
la inscripción en un tema, subir propuestas su memoria y su informe final del curso
Introducción al Trabajo de Título. Además se puede ver su ficha personal, que incluye
la etapa actual del trabajo de título, el tema del trabajo de título junto a su guía.
Además, se puede descargar la propuesta de memoria y el informe final de Introducción al
Trabajo de Título, en caso de haber subido.

\begin{figure}[ht]
    \centering
    \includegraphics[width=0.9\linewidth]{imagenes/titulacion_temas.png}
    \caption{Listado de temas de trabajo de título en el sistema de titulación del DCC.}
    \label{fig:titulacion_temas}
\end{figure}

Si se es profesor, se pueden publicar temas de trabajo de título y ver solicitudes de
inscripción de estudiantes y ver a los memoristas que se está guiando junto a sus
respectivas fichas. Si se es coordinador de titulación, se puede ver el listado de todos
los memoristas, sumado a lo que puede ver un profesor.

\newpage

\begin{figure}[ht]
    \centering
    \includegraphics[width=0.9\linewidth]{imagenes/titulacion_ficha.png}
    \caption{Ficha de un estudiante en el sistema de titulación del DCC.}
    \label{fig:titulacion_ficha}
\end{figure}

Actualmente, las funcionalidades de este sistema están enfocadas en el ramo Introducción
al Trabajo de Título, por lo que no cuenta con las funcionalidades de asignar miembros de
comisiones examinadoras ni de visualizar las comisiones asignadas a cada estudiante, que
corresponden a Trabajo de Título.


\section{Sistema de Seguimiento de Memorias}\label{sec:sa:ssm}

El Sistema de Seguimiento de Memorias (SSM), es un sistema desarrollado por Matías Rivas
Aguilera en 2024 \cite{SistemaMonitoreoMemorias} y extendido por Diego Orellana Vidal en
2025 como sus respectivas memorias para optar al título de Ingeniero Civil en Computación
\cite{SMM_2025}. En la figura \ref{fig:smm_principal} se puede apreciar
la vista principal del SSM. Este sistema se encuentra en producción desde julio de 2025
pero no ha tenido uso hasta la fecha de redacción de esta memoria. Este sistema permite
a la jefatura de estudios monitorear los plazos de entrega de los informes finales de
Trabajo de Título y gestionar la corrección de los mismos. Para realizar esta tarea, el
sistema requiere saber cuales son las comisiones examinadoras asignadas a cada trabajo de
título y el método que se utiliza para obtenerlos es que el usuario los ingrese, por lo
que el sistema permite agregar, eliminar y modificar memorias, estudiantes, profesores
y miembros de comisiones examinadoras. El ingreso de esta información es realizado por la
jefatura de estudios, una vez la coordinación de titulación realice la asignación de las
comisiones. Esto sucede a más tardar en la semana académica número 12 de cada semestre.

\begin{figure}[ht]
    \centering
    \includegraphics[width=0.9\linewidth]{imagenes/smm_main.png}
    \caption{Vista principal del SSM que muestra un listado de memoristas.}
    \label{fig:smm_principal}
\end{figure}

El SSM ofrece dos formas de ingresar miembros de comisiones examinadoras. La primera es de
forma directa, llenando un formulario que permite asignar un/a profesor/a a la vez a una
memoria a la que se asigna. La segunda es mediante la subida de un archivo CSV que sirve
para agregar varias memorias simultáneamente, como se muestra en la figura
\ref{fig:smm_csv}. Cada línea del archivo representa una memoria y debe contener
las siguientes columnas:
\begin{multicols}{3}
    \begin{itemize}
        \item Estudiante
        \item Correo Estudiante
        \item Tema
        \item Guías
        \item Correos Guías
        \item Coguías
        \item Correos Coguías
        \item Integrantes
        \item Correos Integrantes
    \end{itemize}
\end{multicols}

\begin{figure}[ht]
    \centering
    \includegraphics[width=0.9\linewidth]{imagenes/smm_csv.png}
    \caption{Formulario de subida de archivo CSV del SSM.}
    \label{fig:smm_csv}
\end{figure}

La principal limitación del SSM respecto a la asignación de comisiones examinadoras es
que fue desarrollado para que sea utilizado por la jefatura de estudios
\cite[p.~3]{SistemaMonitoreoMemorias}. Como la coordinación de titulación es la encargada
de asignar las comisiones, el sistema fue diseñado para facilitar el registro de
comisiones ya definidas y no la asignación de estas.

Este enfoque se puede ver en la asignación de integrantes a una comisión mediante la
subida de un archivo CSV, que permite agregar varias memorias simultáneamente, haciendo
bastante sencillo el registro de comisiones pero no ofrece ayuda alguna para generar el
archivo CSV. Además, el sistema no ofrece ningún tipo de validación sobre las
restricciones de la asignación de comisiones examinadoras, como que cada comisión debe
tener al menos un/a profesor/a con jerarquía de profesor/a.
\cite[p.~17]{ReglamentoEstudios}

En el caso de la asignación de integrantes a una comisión mediante un formulario, también
se puede notar que el SSM no está diseñado para asignar integrantes a comisiones
examinadoras, ya que el formulario no permite asignar varios integrantes a una comisión a
la vez como se puede apreciar en la figura \ref{fig:smm_form}. Como por cada comisión se
requieren al menos 2 integrantes además de los guías, se debe llenar al menos tres veces
el formulario por cada comisión, lo que resulta ineficiente y tedioso.

\begin{figure}[ht]
    \centering
    \includegraphics[width=0.9\linewidth]{imagenes/smm_form.png}
    \caption{Formulario del SSM para agregar a un integrante a la comisión de una
        memoria, seleccionando el profesor deseado y el rol que tendrá.}
    \label{fig:smm_form}
\end{figure}

\section{Sistema de recomendación de comisiones}\label{sec:sa:recomendacion}

El Sistema de recomendación de comisiones \cite{SistemaRecomendacion} es un sistema que
no se encuentra en producción y fue desarrollado por Rodrigo Oportot González en 2024
como su memoria para optar al título de Ingeniero Civil en Computación. Este sistema
propone 7 profesores/as candidatos/as para la comisión examinadora de una memoria,
basándose en el área de conocimiento de los/as profesores/as y el tema del trabajo de
título. Para esto, utiliza procesamiento de lenguaje natural y Machine Learning.

Este sistema es conveniente para la asignación de comisiones examinadoras, ya que ofrece
candidatos/as según su área de conocimiento, permitiendo tener comisiones con mayor
conocimiento en el tema del trabajo de título. No obstante, estas propuestas no toman
en cuenta la cantidad de comisiones que han sido asignadas a cada profesor/a, lo que
puede resultar en profesores/as con mucha carga y que no tengan el tiempo necesario para
examinar las memorias. Además, cae en la misma falta que el SSM, ya que tampoco se tiene
en cuenta restricciones sobre la conformación de comisiones examinadoras, como que cada
comisión debe tener al menos un/a profesor/a con jerarquía de profesor/a.
\cite[p.~17]{ReglamentoEstudios}


\section{Sistema de Asignación de Comisiones}\label{sec:sa:asignacion}

El Sistema de Asignación de Comisiones es un sistema piloto que fue desarrollado por un
equipo del curso CC5401 Ingeniería de Software II del DCC durante el semestre de otoño de 2025.
Los integrantes de este equipo fueron Daniel Sarazua Y., Ignacio Silva Ghisolfo, J. Andreu Díaz P.,
Manuel Saavedra Soto, Monserrat Montero y Ricardo Fernández Reyes.
El piloto trata de una primera versión de una extensión del sistema de titulación del DCC que
agrega un módulo para asignar comisiones examinadoras para los trabajos de título. La
interfaz principal de este módulo lista los estudiantes que están cursando el ramo Trabajo
de Título del periodo académico seleccionado. Cada fila indica el título de una memoria,
el nombre del estudiante, el nombre del guía, el nombre del coguía si es que se tiene y
los miembros de la comisión examinadora si estos fueron asignados. Al final de cada fila
se encuentra un botón que permite agregar una comisión examinadora en caso de que no
se le haya asignado una. Esta interfaz se puede apreciar en la figura
\ref{fig:ingsoft_comisiones}. Sobre la tabla, a la izquierda se encuentra una barra de
búsqueda que permite buscar filas por el contenido de cualquiera de sus campos.

\begin{figure}[ht]
    \centering
    \includegraphics[width=0.85\linewidth]{imagenes/ingsoft_comisiones.png}
    \caption{Interfaz principal del Sistema de Asignación de Comisiones.}
    \label{fig:ingsoft_comisiones}
\end{figure}

Sobre la barra de búsqueda se encuentran los botones de \textit{Sincronizar} y
\textit{Exportar}. El botón \textit{Sincronizar} actualiza el listado de estudiantes que
cursaron o se encuentran cursando el ramo Trabajo de Título en el periodo académico
seleccionado. Para lograr esto, primero se obtienen a los integrantes del ramo desde la
API de UCampus. Luego, se obtiene el tema de cada estudiante desde el sistema de
titulación, buscando el último registro de aprobación de Introducción al Trabajo de
Título asociado al estudiante. La funcionalidad de este botón es útil, pero debe hacerse
manualmente. Además, fuera de este botón, no hay otro mecanismo en el sistema de
titulación para obtener a los integrantes del ramo Trabajo de Título. El botón
\textit{Exportar} permite exportar la lista de comisiones asignadas a un archivo CSV.
Este tiene las siguientes columnas:

\begin{multicols}{3}
    \begin{itemize}
        \item Título
        \item Periodo
        \item Estudiante
        \item Guía
        \item Coguía
        \item Estado
        \item Evaluadores
        \item Correos
    \end{itemize}
\end{multicols}

El archivo exportado tiene el potencial de ser utilizado para importar comisiones
examinadoras asignadas al SSM, ya que contiene toda la información requerida por este
sistema e incluso más, sin embargo, las columnas tienen distintos nombres y la
información se encuentra en otro formato. Por ejemplo, el archivo CSV generado tiene el
caracter \texttt{;} (punto y coma) como separador, mientras que el SSM requiere que ese
caracter sea utilizado para separar los nombres y los correos de los integrantes de la
comisión examinadora en las columnas \textit{Integrantes} y \textit{Correos Integrantes},
respectivamente.

Este sistema fue diseñado para ser utilizado por la coordinación de titulación del DCC,
por lo que al agregar o editar comisiones toma en cuenta elementos como la cantidad de
comisiones a las que ha sido asignada cada profesor/a y que se puedan agregar varios
miembros a una comisión en un mismo formulario. Además, en el formulario se permite
buscar profesores por nombre y área de conocimiento, como se puede ver en la figura
\ref{subfig:ingsoft_buscar}.

\begin{figure}[ht]
    \begin{subfigure}[b]{0.4\textwidth}
        \centering
        \includegraphics[width=\linewidth]{imagenes/ingsoft_agregar_miembros.png}
        \caption{Formulario que permite agregar varios miembros a una comisión examinadora.}
        \label{subfig:ingsoft_agregar}
    \end{subfigure}
    \hfill
    \begin{subfigure}[b]{0.4\textwidth}
        \centering
        \includegraphics[width=\linewidth]{imagenes/ingsoft_buscar_miembros.png}
        \caption{Selector que permite buscar profesores por nombre y área de conocimiento.}
        \label{subfig:ingsoft_buscar}
    \end{subfigure}
    \caption{Formulario de asignación de comisiones examinadoras en el Sistema de Asignación de Comisiones.}
    \label{fig:ingsoft}
\end{figure}

Como se mencionó anteriormente, el Sistema de Asignación de Comisiones es un sistema
piloto, por lo que no tiene todas las características deseadas. En primer lugar, faltan
validaciones al momento de asignar miembros a una comisión examinadora, haciendo posible
asignar a los profesores guía y coguía como integrantes de la misma comisión examinadora,
quedando registrados dos veces en la comisión. En segundo lugar, no se encuentra bien
integrado con el sistema de titulación, sobre el cual fue desarrollado. Esto se debe a
que el sistema permite asignar comisiones examinadoras a memoristas y exportarlas, pero
las comisiones asignadas no son visibles en otros módulos del sistema de titulación, en
los que sería deseable verlas, como por ejemplo las fichas de los estudiantes. En tercer
lugar, el sistema no se encuentra directamente integrado con el SSM.