\begin{resumen}
  La asignación de comisiones examinadoras para los trabajos de título es un proceso
  con importancia, pues son estas las que evalúan los trabajos de título y
  las defensas de los mismos, siendo un factor clave en el éxito de la titulación de un estudiante.
  Además, al realizar la asignación se debe tener en cuenta que los integrantes de las comisiones
  examinadoras cumplan con tener conocimiento en el área del tema del trabajo de título y que tengan
  tiempo disponible para evaluarlo. Por lo tanto, la elección de integrantes de las comisiones
  examinadoras es una tarea de complejidad no menor. Actualmente, esta tarea es realizada con
  una planilla Excel, lo que resulta en un proceso tedioso, propenso a errores y poco eficiente,
  más aún con el constante aumento de estudiantes en el Departamento de Ciencias de la Computación
  (DCC).

  En este contexto, se propone desarrollar y desplegar una herramienta que permita asignar las
  comisiones examinadoras de manera interactiva y eficiente, y que sea capaz de integrarse con
  los sistemas existentes relacionados con el proceso de titulación del DCC, como el Sistema de
  Titulación y el Sistema de Seguimiento de Memorias (SSM). La solución fue implementada usando
  Django como framework web y PostgreSQL como base de datos. Esta corresponde a una extensión del
  Sistema de Titulación del DCC, que agrega un módulo para asignar comisiones examinadoras a los
  trabajos de título, tomando como base un sistema piloto desarrollado por un equipo del curso
  CC5401 Ingeniería de Software II en el semestre de otoño de 2025. Al piloto se le agregaron funcionalidades como
  validaciones al momento de asignar una comisión, un gráfico de carga de los evaluadores para distribuir
  a los evaluadores de manera equitativa y que pueda exportar las comisiones asignadas en formato CSV y
  también de forma directa al SSM. Para que el SSM pueda recibir las comisiones asignadas de forma directa,
  se implementó una API que permite recibir las comisiones asignadas en formato JSON.
  Tanto la extensión del Sistema de Titulación como la adaptación del SSM fueron desplegadas primero
  en un servidor de test para evaluar su funcionalidad, para luego ser desplegadas en el servidor de
  producción del DCC.

  Para evaluar la solución implementada, se utilizó el módulo de comisiones examinadoras
  para asignar las comisiones examinadoras a los trabajos de título del semestre de primavera 2025
  del DCC. La evaluación mostró que aunque la herramienta requería de algunas mejoras, como optimizaciones en controladores,
  asignar las comisiones con ella significó menos trabajo y tomó menos tiempo que con la planilla Excel.
  El proyecto fue desplegado en el servidor de producción del DCC y permitirá que la asignación de comisiones examinadoras
  no signifique un gasto de tiempo y esfuerzo innecesario para la coordinación de titulación.
\end{resumen}