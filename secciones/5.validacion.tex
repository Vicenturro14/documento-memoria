\chapter{Validación}\label{cap:validacion}
En este capítulo se describe la evaluación de la herramienta de asignación de comisiones
examinadoras. Primero, se explica el método utilizado para la evaluación y luego se
comentan los resultados obtenidos.

\section{Evaluación}
Para establecer el método de evaluación, primero es necesario repasar el objetivo
definido en la sección \ref{sec:intro:objetivos}. Este es desarrollar y desplegar
una herramienta que permita asignar comisiones examinadoras de estudiantes de Trabajo de
Título de manera eficiente. Esto incluye que la herramienta se comunique con los sistemas
existentes relacionados con el proceso de titulación. Por lo tanto, para evaluar la
herramienta se requiere poner a prueba que la asignación de comisiones examinadoras
con el software implementado sea más eficiente que el proceso actual, es decir,
asignar comisiones examinadoras usando una planilla de Excel.

Teniendo en cuenta el objetivo mencionado, se decidió que el método para evaluar
la herramienta sea que la coordinadora de titulación asigne
las comisiones examinadoras de los estudiantes de Trabajo de Título del semestre
primavera 2025 con el software desarrollado desplegado en un servidor de test del DCC,
ya que esto permite evaluar la herramienta en un entorno cercano al proceso real, con
datos reales, con la diferencia de que el software se encuentra desplegado en un servidor
de test del DCC, en vez de uno de producción. Los datos utilizados de estudiantes cursando
Trabajo de Título en el semestre primavera 2025 fueron obtenidos desde la API de UCampus al
momento de la evaluación, mientras que el resto de datos, como los profesores, las memorias
de Introducción al Trabajo de Título y los temas, fueron obtenidos desde un backup al
sistema de titulación realizado una semana antes de la evaluación.
Respecto a las personas que participarán de la evaluación, se consideró solo a la
coordinadora de titulación, ya que al ser la única encargada de asignar comisiones
examinadoras y única usuaria del módulo de asignación de comisiones examinadoras,
es la única persona que puede validar la herramienta y percibir su eficiencia.
Por lo tanto, los resultados obtenidos serán más realistas, tanto en la eficiencia
percibida por la coordinadora, como en los posibles errores que puedan ocurrir.

La asignación de comisiones examinadoras realizada en el servidor de test puede ser
utilizada como asignación oficial de comisiones examinadoras, dado que las comisiones
asignadas en la evaluación pueden ser exportadas a través de la descarga de un archivo
CSV y ser revisadas por la coordinadora de titulación antes de comunicarlas oficialmente
a la coordinación de estudios.

Dos puntos importantes a mencionar son que la evaluación se realizó una semana antes
de la fecha límite de asignación, para que la coordinadora de titulación tuviera tiempo
para realizar la asignación de comisiones examinadoras con el método anterior, en caso de
que hubiera sido inviable asignar las comisiones con el software implementado. Segundo,
al momento de la evaluación, no se había implementado la exportación de comisiones al SSM
por medio de la API, por lo que esto no se evaluó.


\section{Resultados}
La asignación de comisiones examinadoras con el software implementado fue realizada con éxito,
ya que la coordinadora fue capaz de asignar las comisiones y exportarlas a un archivo CSV,
siendo este archivo utilizado oficialmente para comunicar las comisiones examinadoras
a la coordinación de estudios. La coordinadora dejó comentarios sobre los aspectos positivos
y los que creía que se debían mejorar.

En cuanto a los aspectos positivos, la coordinadora mencionó que el proceso fue más rápido
que el proceso manual, y por lo tanto, más eficiente. También mencionó que el gráfico de
carga fue de mucha ayuda para visualizar la carga de los evaluadores y así poder distribuir
a los evaluadores en las comisiones sin sobrecargarlos. Por último, mencionó que fue útil
que la barra de búsqueda permitiera buscar evaluadores por nombre y por área de conocimiento.

En cuanto a los aspectos que creía que se debían mejorar, la coordinadora mencionó que en
la lista de alumnos a quienes se debe asignar comisión no se muestra coguía ni coguía externo.
También mencionó que después de asignar una comisión, la página se recarga y el listado de
estudiantes se muestra desde el principio, por lo que cada vez tuvo que hacer scroll para
proceder con la siguiente asignación. Otra observación fue que al seleccionar un alumno
para asignarle una comisión, el modal tarda alrededor de 13 segundos en desplegarse.
Por último, indicó que el gráfico de carga tiene mucho margen superior, perdiendo espacio
que podría usarse para ampliar el gráfico.


\section{Discusión de resultados}

Los comentarios positivos de la coordinadora de titulación indican que las funcionalidades
implementadas como el gráfico de carga y las barras de búsqueda fueron útiles para
realizar la asignación de comisiones examinadoras de manera más eficiente, ya que
dan acceso rápido al usuario a información relevante para la asignación de comisiones.
Por otro lado, a excepción de la falta de coguía y coguía externo, los aspectos que se deben
mejorar no son problemas graves que impidan la utilización de la herramienta, sino aspectos
que pueden mejorar la experiencia del usuario y la eficiencia de la asignación de comisiones
examinadoras.

En general, la evaluación del módulo de asignación de comisiones examinadoras se puede considerar
exitosa, ya que cumple con el objetivo de implementar una herramienta que permita asignar
comisiones examinadoras de manera eficiente, puesto que realizar la asignación con la
implementación realizada permite completar el proceso más rápido que con el método manual.
Incluso teniendo en cuenta la demora de aparición del modal de asignación de comisión y el scroll
que se debe hacer para proceder con la siguiente asignación, el proceso es más rápido que el método
anterior.

\section{Correcciones}
Dados los comentarios de la coordinadora de titulación, se realizaron correcciones a la herramienta
para mejorar la experiencia del usuario y la eficiencia de la asignación de comisiones examinadoras.
Primero, se corrigió la falta de coguía y coguía externo de forma inmediata, ya que la falta de estos
se debía a un error de tipografía en el template de la tabla de asignación de comisiones examinadoras.
Posteriormente, se redujo el margen superior del gráfico de carga, lo que permitió ampliar el gráfico
verticalmente.

Respecto a la demora de aparición del modal de asignación de comisión, se optimizó el código del
controlador asociado al modal, precalculando la cadena de texto con las áreas de conocimiento
de cada evaluador y evitando consultar por todos los evaluadores cada vez que se abre el modal.
Por último, se eliminó la recarga de la página al asignar una comisión, lo que permitió quitar
el scroll que se debe hacer para proceder con la siguiente asignación. El razonamiento y la justificación
de la corrección de los últimos dos aspectos puede encontrarse en la sección \ref{subsec:cambios_retroalimentacion}.



