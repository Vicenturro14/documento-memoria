\chapter{Introducción}

\section{Contexto}
El hito final de una carrera en la Facultad de Ciencias Físicas y Matemáticas de la
Universidad de Chile, como lo es Ingeniería Civil en Computación, es el Trabajo de
Titulación. Este es un proceso que se divide en tres etapas: el curso de Introducción al
Trabajo de Título, el curso de Trabajo de Título y el Examen de Título. La coordinación
de titulación del departamento debe asignar una comisión examinadora a cada estudiante
cursando Trabajo de Título y enviarlas a la coordinación de estudios del departamento con
plazo máximo la semana número 12 de cada semestre. Cada comisión tiene el rol de evaluar
el informe redactado por el estudiante en el curso de Trabajo de Título y la defensa en el
examen de título. Cada comisión está compuesta por el profesor guía, el profesor coguía,
en caso de tener, y al menos dos integrantes más. Estos integrantes adicionales pueden
ser académicos/as de la Facultad de Ciencias Físicas
y Matemáticas de la Universidad de Chile (FCFM) o profesores expertos externos, con la
condición de que al menos uno debe ser académico/a de la FCFM con jerarquía de
profesor/a. \cite[p.~17]{ReglamentoEstudios}

Los/as estudiantes requieren aprobar la defensa de su trabajo de título para obtener el
título \cite[pp.~15-16]{ReglamentoEstudios}, por lo que es crucial que la comisión
examinadora sea capaz de evaluar el trabajo de título y su defensa de manera adecuada.
Esto involucra factores como que los/as profesores/as tengan conocimiento en el área del
tema del trabajo de título y que tengan tiempo disponible para evaluarlo. Por lo tanto,
la elección de integrantes de las comisiones examinadoras es una tarea de complejidad
no menor.

En el Departamento de Ciencias de la Computación de la Universidad de Chile (DCC), la
selección de integrantes de las comisiones examinadoras era realizada de forma manual y el
registro de estos con hojas de cálculo tipo Excel por el coordinador de titulación. Debido al
gran aumento de estudiantes en el DCC en los últimos años, el número de memoristas también
ha crecido, haciendo que la asignación manual de comisiones sea una tarea tediosa,
propensa a errores y poco eficiente. A modo de ejemplo del aumento de estudiantes, en el
semestre de primavera de 2024 había un total de 60 inscritos en los cursos de Trabajo de
Título y en el semestre de otoño de 2025 aumentó a 103, según el catálogo de cursos de
UCampus de la FCFM \cite{EstudiantesTrabajoTitulo20242,EstudiantesTrabajoTitulo20251}.

Es por lo anterior que apareció la necesidad de una herramienta que apoye la
asignación de comisiones examinadoras, aportando información sobre las áreas de
conocimiento de los/as profesores/as y la cantidad de comisiones que han sido asignadas
a cada profesor/a, y que se encuentre integrada con las herramientas ya existentes, como
el Sistema de Titulación.

\section{Objetivos}\label{sec:intro:objetivos}

\subsection{Objetivo General}
El objetivo de esta memoria fue desarrollar y desplegar una herramienta que permita que la
asignación de comisiones examinadoras para las memorias sea una tarea eficiente.
Esto incluye que la herramienta se comunique con los sistemas existentes relacionados con
el proceso de titulación, como el Sistema de Titulación y el Sistema de Seguimiento de
Memorias (SSM).

\subsection{Objetivos Específicos}
Para cumplir con el objetivo, se plantearon los siguientes objetivos específicos:
\begin{itemize}
  \item Desarrollar una herramienta que permita asignar comisiones de forma interactiva.
  \item Integrar la herramienta con los sistemas existentes relacionados con el proceso de
        titulación, en particular con el Sistema de Titulación y el Sistema de Seguimiento
        de Memorias (SSM).
  \item Dejar la herramienta desarrollada en producción.
\end{itemize}

\section{Solución Propuesta}
A modo general, la solución propuesta consiste en cuatro partes. La primera parte
corresponde a la adición de funcionalidades en el módulo de asignación de comisiones
examinadoras, la segunda es la integración del módulo con el Sistema de Titulación, la
tercera es la integración con el Sistema de Seguimiento de Memorias (SSM) y la cuarta es
el despliegue de la solución en los servidores del DCC.

\subsection{Desarrollo del módulo de asignación de comisiones examinadoras}
La primera funcionalidad que se agregó al módulo de asignación de comisiones
examinadoras es la validación de las comisiones asignadas. Se verifica que las
comisiones tengan al menos dos integrantes además de los guías. De estos integrantes,
al menos uno debe tener jerarquía de profesor, es decir, que sean académicos de jornada
completa (AJC) o académicos de jornada parcial (AJP). También se verifica que no
se repitan académicos en la misma comisión.

Luego, se implementó un filtro en la interfaz principal del módulo que permite
mostrar todas las memorias, solo las memorias con comisión completamente asignada o solo
las memorias con comisión incompleta. Este filtro permite navegar con mayor facilidad
entre las memorias al momento de asignar comisiones examinadoras.

Además, en la misma interfaz se agregó un gráfico que muestra la carga de los
académicos. Específicamente, por cada académico se muestra la cantidad de trabajos de título
que guía y la cantidad de comisiones que integra. El gráfico es de barras y ordena
a los académicos de forma decreciente por la carga que tienen. De esta forma,
se podrá identificar académicos con carga excesiva y, por lo tanto, evitar que se les
siga sobrecargando con comisiones adicionales.


\subsection{Integración en el Sistema de Titulación}
Previo a este trabajo, el Sistema de Titulación solo contaba con un listado de estudiantes de
Introducción al Trabajo de Título, por lo que se agregó a los estudiantes de
Trabajo de Título para integrar correctamente el módulo de asignación de comisiones
examinadoras. Para esto, se agregó una pestaña con un listado de integrantes de Trabajo
de Título en los distintos periodos académicos. Para mantener este listado actualizado,
se implementó la posibilidad de actualizar el listado con datos de UCampus.

Luego, se agregó la posibilidad de exportar las comisiones examinadoras asignadas, que
se puede realizar de dos formas. Por un lado, se puede descargar un archivo CSV con las
comisiones, y por otro lado, estas pueden ser exportadas directamente hacia el Sistema de
Seguimiento de Memorias, mediante su API REST.


\subsection{Integración con el Sistema de Seguimiento de Memorias}
El Sistema de Seguimiento de Memorias (SSM) es un software utilizado por la coordinación de
estudios para monitorear las fechas de entrega de los informes de Trabajo de Título y
coordinar la corrección de estos documentos con los integrantes de la comisión examinadora
respectiva. Antes de esta memoria, para que el SSM pudiera obtener las memorias y las
comisiones asignadas a cada una, se debía subir manualmente un archivo CSV.

Para hacer la exportación de las comisiones examinadoras al Sistema de Seguimiento de
Memorias con mayor facilidad, se implementó en el Sistema de Titulación la opción de exportar
las comisiones examinadoras asignadas usando el protocolo HTTP. Además, se le creó al Sistema de
Seguimiento de Memorias una API REST de un solo endpoint para importar las comisiones examinadoras
asignadas desde el Sistema de Titulación.

\subsection{Despliegue de la solución}\label{sec:intro:despliegue}
Por último, la solución desarrollada fue desplegada en los servidores del DCC para que
pueda ser utilizada por la coordinación de titulación. Para este despliegue, se
utilizaron contenedores de Docker, uno para la aplicación de Django y otro para la base
de datos de PostgreSQL.