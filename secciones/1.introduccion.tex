\chapter{Introducción}

\section{Contexto}
El hito final de una carrera en la Facultad de Ciencias Físicas y Matemáticas de la
Universidad de Chile, como lo es Ingeniería Civil en Computación, es el Trabajo de
Titulación. Este es un proceso que se divide en tres etapas: el curso de Introducción al
Trabajo de Título, el curso de Trabajo de Título y el Examen de Título. La coordinación
de titulación del departamento debe asignar una comisión examinadora a cada estudiante
cursando Trabajo de Título y enviarlas a la jefatura de estudios del departamento con
plazo máximo la semana número 12 de cada semestre. Esta tiene el rol de evaluar el
informe redactado por el estudiante en el curso de Trabajo de Título y la defensa en el
examen de título. Cada comisión está compuesta por el profesor guía, el profesor coguía,
en caso de tener, y al menos dos integrantes más. Estos integrantes adicionales pueden
ser académicos/as de la Facultad de Ciencias Físicas
y Matemáticas de la Universidad de Chile (FCFM) o profesores expertos externos, con la
condición de que al menos uno debe ser académico/a de la FCFM con jerarquía de
profesor/a. \cite[p.~17]{ReglamentoEstudios}

Los/as estudiantes requieren aprobar la defensa de su trabajo de título para obtener el
título \cite[pp.~15-16]{ReglamentoEstudios}, por lo que es crucial que la comisión
examinadora sea capaz de evaluar el trabajo de título y su defensa de manera adecuada.
Esto involucra factores como que los/as profesores/as tengan conocimiento en el área del
tema del trabajo de título y que tengan tiempo disponible para evaluarlo. Por lo tanto,
la elección de integrantes de las comisiones examinadoras es una tarea de complejidad
no menor.

En el Departamento de Ciencias de la Computación de la Universidad de Chile (DCC), la
selección de integrantes de las comisiones examinadoras es realizada de forma manual y el
registro de estos con hojas de cálculo tipo Excel. Debido al gran aumento de estudiantes
en el DCC en los últimos años, el número de memoristas también ha crecido, haciendo que
la asignación manual de comisiones sea una tarea tediosa, propensa a errores y poco
eficiente. A modo de ejemplo del aumento de estudiantes, en el semestre de primavera de
2024 había un total de 60 inscritos en los cursos de Trabajo de Título y en el semestre
de otoño de 2025 aumentó a 103, según el catálogo de cursos de UCampus de la FCFM
\cite{EstudiantesTrabajoTitulo20242,EstudiantesTrabajoTitulo20251}. A esto se le suma que
no hay una base de datos 100\% correcta de estudiantes que requieran
comisiones.

Es por lo anterior que aparece la necesidad de una herramienta que permita facilitar la
asignación de comisiones examinadoras, aportando información sobre las áreas de
conocimiento de los/as profesores/as y la cantidad de comisiones que han sido asignadas
a cada profesor/a, y que se encuentre integrada con las herramientas ya existentes, como
el sistema de titulación.

\section{Objetivos}

\subsection{Objetivo General}
El objetivo de esta memoria es desarrollar y desplegar una herramienta que permita que la
asignación de comisiones examinadoras para las memorias sea una tarea eficiente.
Esto incluye que la herramienta se comunique con los sistemas existentes relacionados con
el proceso de titulación, como el Sistema de Titulación y el Sistema de Seguimiento de
Memorias (SSM).

\subsection{Objetivos Específicos}
Para cumplir con el objetivo, se plantean los siguientes objetivos específicos:
\begin{itemize}
    \item Desarrollar una herramienta que permita asignar comisiones de forma interactiva.
    \item Integrar la herramienta con los sistemas existentes relacionados con el proceso de
          titulación, en particular con el sistema de titulación y el Sistema de Seguimiento de
          Memorias (SSM).
    \item Dejar la herramienta desarrollada en producción.
\end{itemize}

\section{Solución Propuesta}
Considerando los objetivos mencionados, se propone realizar una extensión del sistema de
asignación de comisiones examinadoras desarrollado por un equipo del curso CC5401
Ingeniería de Software II. Como se mencionó en la sección \ref{sec:sa:asignacion}, el
sistema de asignación de comisiones examinadoras es un sistema piloto y una extensión del
sistema de titulación del DCC, que agrega un módulo para asignar comisiones examinadoras
a los memoristas. Dado que el sistema de titulación fue desarrollado en Python con el
framework web Django y la base de datos relacional PostgreSQL, la extensión del sistema
de asignación se realizará utilizando las mismas tecnologías.

A modo general, la solución propuesta consiste en cuatro partes. La primera parte
corresponde a la adición de funcionalidades en el módulo de asignación de comisiones
examinadoras, la segunda es la integración del módulo con el Sistema de Titulación, la
tercera es la integración con el Sistema de Seguimiento de Memorias (SSM) y la cuarta es
el despliegue de la solución en los servidores del DCC.

\subsection{Desarrollo del módulo de asignación de comisiones examinadoras}
La primera funcionalidad que se agregará al módulo de asignación de comisiones
examinadoras es la validación de las comisiones asignadas. Se debe verificar que las
comisiones tengan al menos dos integrantes además de los guías. De estos integrantes,
al menos uno debe tener jerarquía de profesor, es decir, que sean académicos de jornada
completa (AJC) o académicos de jornada parcial (AJP). También se debe verificar que no
se repitan académicos en la misma comisión.

Luego, se implementará un filtro en la interfaz principal del módulo que permita
mostrar todas las memorias, solo las memorias con comisión completamente asignada o solo
las memorias con comisión incompleta. Este filtro permitirá navegar con mayor facilidad
entre las memorias al momento de asignar comisiones examinadoras.

Además, en la misma interfaz se agregará un gráfico que muestre la carga de los
académicos. Específicamente, por cada académico se mostrará la cantidad de comisiones
que guía y la cantidad de comisiones que integra. El gráfico será de columnas y mostrará
a los académicos ordenados por la carga que tienen. De esta forma, se podrá identificar
académicos con carga excesiva y, por lo tanto, evitar que se les asigne una comisión
examinadora.


\subsection{Integración en el Sistema de Titulación}
Actualmente el sistema de titulación solo cuenta con un listado de estudiantes de
Introducción al Trabajo de Título, por lo que se debe agregar a los estudiantes de
Trabajo de Título para integrar correctamente el módulo de asignación de comisiones
examinadoras. Para esto, se agregará una pestaña con un listado de integrantes de Trabajo
de Título en los distintos periodos académicos. Para mantener este listado actualizado,
se implementará un cronjob que se encargue de obtener periódicamente desde UCampus a los
estudiantes que están cursando Trabajo de Título.

Después, se agregará un botón de publicación de comisiones. Este botón confirmará que
coordinación de titulación terminó de asignar las comisiones examinadoras de un periodo
académico en particular y que estas pueden ser utilizadas por otros sistemas. Cuando se
presione este botón se solicitará confirmar la publicación indicando adicionalmente
la cantidad de comisiones asignadas. Este botón se ubicará en la interfaz principal del
módulo de asignación de comisiones examinadoras junto a los botones de \textit{Sincronizar}
y \textit{Exportar}. Para evitar confusiones entre los botones de publicación y exportación,
el botón de exportación \textit{Exportar}, que sirve para descargar un archivo CSV con las
comisiones examinadoras asignadas, se le cambiará el texto a \textit{Descargar}.

Luego, se agregará la comisión examinadora a la ficha de cada estudiante, con el objetivo
de que los estudiantes puedan saber quiénes son los miembros de su comisión examinadora,
una vez esta sea publicada.

\subsection{Integración con el Sistema de Seguimiento de Memorias}
Para integrar las comisiones examinadoras en el Sistema de Seguimiento de Memorias, primero
se cambiará el archivo CSV generado por el módulo de asignación de comisiones examinadoras
para que tenga las columnas y el formato solicitados por el SSM.

Para hacer la exportación de las comisiones examinadoras al Sistema de Seguimiento de
Memorias más fácilmente, se implementará un endpoint en el sistema de titulación para
obtener las comisiones examinadoras asignadas por una API REST. Además, desde el
Sistema de Seguimiento de Memorias se agregará la opción de importar las comisiones
desde esta API.

\subsection{Despliegue de la solución}
Por último, la solución desarrollada se desplegará en los servidores del DCC para que
pueda ser utilizada por la coordinación de titulación. Para este despliegue, se
utilizarán contenedores de Docker, uno para la aplicación de Django y otro para la base
de datos de PostgreSQL.