% se puede agregar la opción [english] para 
%  memorias o tesis en inglés (borrando el archivo .aux)
\documentclass{umemoria} 

\depto{Departamento de Ciencias de la Computación}
\author{Vicente Esteban Olivares Gómez}
\title{MÓDULO DE ASIGNACIÓN DE COMISIONES EXAMINADOREAS EN EL SISTEMA DE TITULACIÓN DCC}

% incluir ambos comandos para una doble titulación
%  o quitar el comando que no aplica
\memoria{Ingeniero Civil en Computación}
% \tesis{Magíster en ???}
%\tesis{Doctor en ???} % incluir solo este comando para doctorados

% puede haber varios profesores guía seperados por coma;
% pero si es una memoria, solo puede haber un profesor guía
\guia{María Cecilia Bastarrica Pineyro} 

% puede haber varios profesores co-guía seperados por coma;
% pero si es una memoria, el profesor co-guía será el primer
% integrante de la comisión
%\coguia{Nombre Completo Co-Guía} % incluir en caso de co-guía de *tesis*

%\cotutela{Nombre Institución} % incluir en caso de cotutela
\comision{César Ramón Guerrero Saldivia, Francisco Javier Gutiérrez Figueroa}

%\auspicio{Nombre Institución} % incluir en caso de recibir financiamiento

% tiene que ser el año en que se da el examen de título/grado (defensa)
%\anho{2021} % incluir solo para reemplazar el año actual

\usepackage{lipsum}
\usepackage{url}
\usepackage{subcaption}
\usepackage{multicol}

\begin{document}

\frontmatter
\maketitle

\begin{resumen}
    \lipsum[1-4]
\end{resumen}

% opcional: incluir para tesis en inglés;
%  en este caso hay que tener el resumen y abstract
%   en ambos idiomas
%\begin{abstract}
%\lipsum[1-4]
%\end{abstract}

\begin{dedicatoria}
    Una dedicatoria corta.
\end{dedicatoria}

\begin{thanks}
    \lipsum[1-2]
\end{thanks}

\tableofcontents
% \listoftables % opcional
\listoffigures % opcional

\mainmatter

\chapter{Introducción}

\section{Contexto}
El hito final de una carrera en la Facultad de Ciencias Físicas y Matemáticas de la
Universidad de Chile, como lo es Ingeniería Civil en Computación, es el Trabajo de
Titulación. Este es un proceso que se divide en tres etapas: el curso de Introducción al
Trabajo de Título, el curso de Trabajo de Título y el Examen de Título. La coordinación
de titulación del departamento debe asignar una comisión examinadora a cada estudiante
cursando Trabajo de Título y enviarlas a la coordinación de estudios del departamento con
plazo máximo la semana número 12 de cada semestre. Cada comisión tiene el rol de evaluar
el informe redactado por el estudiante en el curso de Trabajo de Título y la defensa en el
examen de título. Cada comisión está compuesta por el profesor guía, el profesor coguía,
en caso de tener, y al menos dos integrantes más. Estos integrantes adicionales pueden
ser académicos/as de la Facultad de Ciencias Físicas
y Matemáticas de la Universidad de Chile (FCFM) o profesores expertos externos, con la
condición de que al menos uno debe ser académico/a de la FCFM con jerarquía de
profesor/a. \cite[p.~17]{ReglamentoEstudios}

Los/as estudiantes requieren aprobar la defensa de su trabajo de título para obtener el
título \cite[pp.~15-16]{ReglamentoEstudios}, por lo que es crucial que la comisión
examinadora sea capaz de evaluar el trabajo de título y su defensa de manera adecuada.
Esto involucra factores como que los/as profesores/as tengan conocimiento en el área del
tema del trabajo de título y que tengan tiempo disponible para evaluarlo. Por lo tanto,
la elección de integrantes de las comisiones examinadoras es una tarea de complejidad
no menor.

En el Departamento de Ciencias de la Computación de la Universidad de Chile (DCC), la
selección de integrantes de las comisiones examinadoras es realizada de forma manual y el
registro de estos con hojas de cálculo tipo Excel por el coordinador de titulación. Debido al
gran aumento de estudiantes en el DCC en los últimos años, el número de memoristas también
ha crecido, haciendo que la asignación manual de comisiones sea una tarea tediosa,
propensa a errores y poco eficiente. A modo de ejemplo del aumento de estudiantes, en el
semestre de primavera de 2024 había un total de 60 inscritos en los cursos de Trabajo de
Título y en el semestre de otoño de 2025 aumentó a 103, según el catálogo de cursos de
UCampus de la FCFM \cite{EstudiantesTrabajoTitulo20242,EstudiantesTrabajoTitulo20251}.

Es por lo anterior que aparece la necesidad de una herramienta que apoye la
asignación de comisiones examinadoras, aportando información sobre las áreas de
conocimiento de los/as profesores/as y la cantidad de comisiones que han sido asignadas
a cada profesor/a, y que se encuentre integrada con las herramientas ya existentes, como
el Sistema de Titulación.

\section{Objetivos}\label{sec:intro:objetivos}

\subsection{Objetivo General}
El objetivo de esta memoria es desarrollar y desplegar una herramienta que permita que la
asignación de comisiones examinadoras para las memorias sea una tarea eficiente.
Esto incluye que la herramienta se comunique con los sistemas existentes relacionados con
el proceso de titulación, como el Sistema de Titulación y el Sistema de Seguimiento de
Memorias (SSM).

\subsection{Objetivos Específicos}
Para cumplir con el objetivo, se plantean los siguientes objetivos específicos:
\begin{itemize}
  \item Desarrollar una herramienta que permita asignar comisiones de forma interactiva.
  \item Integrar la herramienta con los sistemas existentes relacionados con el proceso de
        titulación, en particular con el Sistema de Titulación y el Sistema de Seguimiento
        de Memorias (SSM).
  \item Dejar la herramienta desarrollada en producción.
\end{itemize}

\section{Solución Propuesta}
A modo general, la solución propuesta consiste en cuatro partes. La primera parte
corresponde a la adición de funcionalidades en el módulo de asignación de comisiones
examinadoras, la segunda es la integración del módulo con el Sistema de Titulación, la
tercera es la integración con el Sistema de Seguimiento de Memorias (SSM) y la cuarta es
el despliegue de la solución en los servidores del DCC.

\subsection{Desarrollo del módulo de asignación de comisiones examinadoras}
La primera funcionalidad que se agregará al módulo de asignación de comisiones
examinadoras es la validación de las comisiones asignadas. Se debe verificar que las
comisiones tengan al menos dos integrantes además de los guías. De estos integrantes,
al menos uno debe tener jerarquía de profesor, es decir, que sean académicos de jornada
completa (AJC) o académicos de jornada parcial (AJP). También se debe verificar que no
se repitan académicos en la misma comisión.

Luego, se implementará un filtro en la interfaz principal del módulo que permita
mostrar todas las memorias, solo las memorias con comisión completamente asignada o solo
las memorias con comisión incompleta. Este filtro permitirá navegar con mayor facilidad
entre las memorias al momento de asignar comisiones examinadoras.

Además, en la misma interfaz se agregará un gráfico que muestre la carga de los
académicos. Específicamente, por cada académico se mostrará la cantidad de comisiones
que guía y la cantidad de comisiones que integra. El gráfico será de barras y mostrará
a los académicos ordenados de forma decreciente  por la carga que tienen. De esta forma,
se podrá identificar académicos con carga excesiva y, por lo tanto, evitar que se les
asigne a más comisiones.


\subsection{Integración en el Sistema de Titulación}
Actualmente el Sistema de Titulación solo cuenta con un listado de estudiantes de
Introducción al Trabajo de Título, por lo que se debe agregar a los estudiantes de
Trabajo de Título para integrar correctamente el módulo de asignación de comisiones
examinadoras. Para esto, se agregará una pestaña con un listado de integrantes de Trabajo
de Título en los distintos periodos académicos. Para mantener este listado actualizado,
se implementará un cronjob que se encargue de obtener periódicamente desde UCampus a los
estudiantes que están cursando Trabajo de Título.

Luego, se agregará la posibilidad de exportar las comisiones examinadoras asignadas, que
se podrá realizar de dos formas. Por un lado, se podrá descargar un archivo CSV con las
comisiones, y por otro lado, estas podrán ser exportadas directamente hacia el Sistema de
Seguimiento de Memorias, mediante su API.


\subsection{Integración con el Sistema de Seguimiento de Memorias}
El Sistema de Seguimiento de Memorias (SSM) es un software utilizado por la coordinación de
estudios para monitorear las fechas de entrega de los informes de Trabajo de Título y
coordinar la corrección de estos documentos con los integrantes de la comisión examinadora
respectiva. Antes de esta memoria, para que el SSM pudiera obtener las memorias y las
comisiones asignadas a cada una, se debía subir manualmente un archivo CSV.

Para hacer la exportación de las comisiones examinadoras al Sistema de Seguimiento de
Memorias más fácilmente, se implementará un endpoint en el Sistema de Titulación para
obtener las comisiones examinadoras asignadas por una API REST. Además, desde el
Sistema de Seguimiento de Memorias se agregará la opción de importar las comisiones
desde esta API.

\subsection{Despliegue de la solución}\label{sec:intro:despliegue}
Por último, la solución desarrollada se desplegará en los servidores del DCC para que
pueda ser utilizada por la coordinación de titulación. Para este despliegue, se
utilizarán contenedores de Docker, uno para la aplicación de Django y otro para la base
de datos de PostgreSQL.
\chapter{Situación Actual}

Actualmente existen varios proyectos y sistemas relacionados con el proceso de titulación
en el DCC. Dentro de estos se encuentran el sistema de titulación del DCC, el Sistema de
Monitoreo de Memorias \cite{SistemaMonitoreoMemorias}, un sistema de recomendación de
comisiones \cite{SistemaRecomendacion} y un sistema de asignación de comisiones
desarrollado por un grupo del curso CC5401 Ingeniería de Software II. A continuación se
describen las características y las limitaciones de estos sistemas, relacionadas con la
asignación de comisiones examinadoras.


\section{Sistema de Titulación DCC}\label{sec:sa:titulacion}

El sistema de titulación del DCC es un sistema web en producción que ofrece distintas
funcionalidades dependiendo de si se es estudiante, profesor o coordinador de titulación.
Como estudiante, se pueden ver un listado de temas para trabajos de título, solicitar
la inscripción en un tema, subir propuestas su memoria y su informe final del curso
Introducción al Trabajo de Título. Además se puede ver su ficha personal, que incluye
la etapa actual del trabajo de título, el tema del trabajo de título junto a su guía.
Además, se puede descargar la propuesta de memoria y el informe final de Introducción al
Trabajo de Título, en caso de haber subido.

\begin{figure}[ht]
    \centering
    \includegraphics[width=0.9\linewidth]{imagenes/titulacion_temas.png}
    \caption{Listado de temas de trabajo de título en el sistema de titulación del DCC.}
    \label{fig:titulacion_temas}
\end{figure}

Si se es profesor, se pueden publicar temas de trabajo de título y ver solicitudes de
inscripción de estudiantes y ver a los memoristas que se está guiando junto a sus
respectivas fichas. Si se es coordinador de titulación, se puede ver el listado de todos
los memoristas, sumado a lo que puede ver un profesor.

\newpage

\begin{figure}[ht]
    \centering
    \includegraphics[width=0.9\linewidth]{imagenes/titulacion_ficha.png}
    \caption{Ficha de un estudiante en el sistema de titulación del DCC.}
    \label{fig:titulacion_ficha}
\end{figure}

Actualmente, las funcionalidades de este sistema están enfocadas en el ramo Introducción
al Trabajo de Título, por lo que no cuenta con las funcionalidades de asignar miembros de
comisiones examinadoras ni de visualizar las comisiones asignadas a cada estudiante, que
corresponden a Trabajo de Título.


\section{Sistema de Seguimiento de Memorias}\label{sec:sa:ssm}

El Sistema de Seguimiento de Memorias (SSM), es un sistema desarrollado por Matías Rivas
Aguilera en 2024 \cite{SistemaMonitoreoMemorias} y extendido por Diego Orellana Vidal en
2025 como sus respectivas memorias para optar al título de Ingeniero Civil en Computación
\cite{SMM_2025}. En la figura \ref{fig:smm_principal} se puede apreciar
la vista principal del SSM. Este sistema se encuentra en producción desde julio de 2025
pero no ha tenido uso hasta la fecha de redacción de esta memoria. Este sistema permite
a la jefatura de estudios monitorear los plazos de entrega de los informes finales de
Trabajo de Título y gestionar la corrección de los mismos. Para realizar esta tarea, el
sistema requiere saber cuales son las comisiones examinadoras asignadas a cada trabajo de
título y el método que se utiliza para obtenerlos es que el usuario los ingrese, por lo
que el sistema permite agregar, eliminar y modificar memorias, estudiantes, profesores
y miembros de comisiones examinadoras. El ingreso de esta información es realizado por la
jefatura de estudios, una vez la coordinación de titulación realice la asignación de las
comisiones. Esto sucede a más tardar en la semana académica número 12 de cada semestre.

\begin{figure}[ht]
    \centering
    \includegraphics[width=0.9\linewidth]{imagenes/smm_main.png}
    \caption{Vista principal del SSM que muestra un listado de memoristas.}
    \label{fig:smm_principal}
\end{figure}

El SSM ofrece dos formas de ingresar miembros de comisiones examinadoras. La primera es de
forma directa, llenando un formulario que permite asignar un/a profesor/a a la vez a una
memoria a la que se asigna. La segunda es mediante la subida de un archivo CSV que sirve
para agregar varias memorias simultáneamente, como se muestra en la figura
\ref{fig:smm_csv}. Cada línea del archivo representa una memoria y debe contener
las siguientes columnas:
\begin{multicols}{3}
    \begin{itemize}
        \item Estudiante
        \item Correo Estudiante
        \item Tema
        \item Guías
        \item Correos Guías
        \item Coguías
        \item Correos Coguías
        \item Integrantes
        \item Correos Integrantes
    \end{itemize}
\end{multicols}

\begin{figure}[ht]
    \centering
    \includegraphics[width=0.9\linewidth]{imagenes/smm_csv.png}
    \caption{Formulario de subida de archivo CSV del SSM.}
    \label{fig:smm_csv}
\end{figure}

La principal limitación del SSM respecto a la asignación de comisiones examinadoras es
que fue desarrollado para que sea utilizado por la jefatura de estudios
\cite[p.~3]{SistemaMonitoreoMemorias}. Como la coordinación de titulación es la encargada
de asignar las comisiones, el sistema fue diseñado para facilitar el registro de
comisiones ya definidas y no la asignación de estas.

Este enfoque se puede ver en la asignación de integrantes a una comisión mediante la
subida de un archivo CSV, que permite agregar varias memorias simultáneamente, haciendo
bastante sencillo el registro de comisiones pero no ofrece ayuda alguna para generar el
archivo CSV. Además, el sistema no ofrece ningún tipo de validación sobre las
restricciones de la asignación de comisiones examinadoras, como que cada comisión debe
tener al menos un/a profesor/a con jerarquía de profesor/a.
\cite[p.~17]{ReglamentoEstudios}

En el caso de la asignación de integrantes a una comisión mediante un formulario, también
se puede notar que el SSM no está diseñado para asignar integrantes a comisiones
examinadoras, ya que el formulario no permite asignar varios integrantes a una comisión a
la vez como se puede apreciar en la figura \ref{fig:smm_form}. Como por cada comisión se
requieren al menos 2 integrantes además de los guías, se debe llenar al menos tres veces
el formulario por cada comisión, lo que resulta ineficiente y tedioso.

\begin{figure}[ht]
    \centering
    \includegraphics[width=0.9\linewidth]{imagenes/smm_form.png}
    \caption{Formulario del SSM para agregar a un integrante a la comisión de una
        memoria, seleccionando el profesor deseado y el rol que tendrá.}
    \label{fig:smm_form}
\end{figure}

\section{Sistema de recomendación de comisiones}\label{sec:sa:recomendacion}

El Sistema de recomendación de comisiones \cite{SistemaRecomendacion} es un sistema que
no se encuentra en producción y fue desarrollado por Rodrigo Oportot González en 2024
como su memoria para optar al título de Ingeniero Civil en Computación. Este sistema
propone 7 profesores/as candidatos/as para la comisión examinadora de una memoria,
basándose en el área de conocimiento de los/as profesores/as y el tema del trabajo de
título. Para esto, utiliza procesamiento de lenguaje natural y Machine Learning.

Este sistema es conveniente para la asignación de comisiones examinadoras, ya que ofrece
candidatos/as según su área de conocimiento, permitiendo tener comisiones con mayor
conocimiento en el tema del trabajo de título. No obstante, estas propuestas no toman
en cuenta la cantidad de comisiones que han sido asignadas a cada profesor/a, lo que
puede resultar en profesores/as con mucha carga y que no tengan el tiempo necesario para
examinar las memorias. Además, cae en la misma falta que el SSM, ya que tampoco se tiene
en cuenta restricciones sobre la conformación de comisiones examinadoras, como que cada
comisión debe tener al menos un/a profesor/a con jerarquía de profesor/a.
\cite[p.~17]{ReglamentoEstudios}


\section{Sistema de Asignación de Comisiones}\label{sec:sa:asignacion}

El Sistema de Asignación de Comisiones es un sistema piloto que fue desarrollado por un
equipo del curso CC5401 Ingeniería de Software II del DCC durante el semestre de otoño de 2025.
Los integrantes de este equipo fueron Daniel Sarazua Y., Ignacio Silva Ghisolfo, J. Andreu Díaz P.,
Manuel Saavedra Soto, Monserrat Montero y Ricardo Fernández Reyes.
El piloto trata de una primera versión de una extensión del sistema de titulación del DCC que
agrega un módulo para asignar comisiones examinadoras para los trabajos de título. La
interfaz principal de este módulo lista los estudiantes que están cursando el ramo Trabajo
de Título del periodo académico seleccionado. Cada fila indica el título de una memoria,
el nombre del estudiante, el nombre del guía, el nombre del coguía si es que se tiene y
los miembros de la comisión examinadora si estos fueron asignados. Al final de cada fila
se encuentra un botón que permite agregar una comisión examinadora en caso de que no
se le haya asignado una. Esta interfaz se puede apreciar en la figura
\ref{fig:ingsoft_comisiones}. Sobre la tabla, a la izquierda se encuentra una barra de
búsqueda que permite buscar filas por el contenido de cualquiera de sus campos.

\begin{figure}[ht]
    \centering
    \includegraphics[width=0.85\linewidth]{imagenes/ingsoft_comisiones.png}
    \caption{Interfaz principal del Sistema de Asignación de Comisiones.}
    \label{fig:ingsoft_comisiones}
\end{figure}

Sobre la barra de búsqueda se encuentran los botones de \textit{Sincronizar} y
\textit{Exportar}. El botón \textit{Sincronizar} actualiza el listado de estudiantes que
cursaron o se encuentran cursando el ramo Trabajo de Título en el periodo académico
seleccionado. Para lograr esto, primero se obtienen a los integrantes del ramo desde la
API de UCampus. Luego, se obtiene el tema de cada estudiante desde el sistema de
titulación, buscando el último registro de aprobación de Introducción al Trabajo de
Título asociado al estudiante. La funcionalidad de este botón es útil, pero debe hacerse
manualmente. Además, fuera de este botón, no hay otro mecanismo en el sistema de
titulación para obtener a los integrantes del ramo Trabajo de Título. El botón
\textit{Exportar} permite exportar la lista de comisiones asignadas a un archivo CSV.
Este tiene las siguientes columnas:

\begin{multicols}{3}
    \begin{itemize}
        \item Título
        \item Periodo
        \item Estudiante
        \item Guía
        \item Coguía
        \item Estado
        \item Evaluadores
        \item Correos
    \end{itemize}
\end{multicols}

El archivo exportado tiene el potencial de ser utilizado para importar comisiones
examinadoras asignadas al SSM, ya que contiene toda la información requerida por este
sistema e incluso más, sin embargo, las columnas tienen distintos nombres y la
información se encuentra en otro formato. Por ejemplo, el archivo CSV generado tiene el
caracter \texttt{;} (punto y coma) como separador, mientras que el SSM requiere que ese
caracter sea utilizado para separar los nombres y los correos de los integrantes de la
comisión examinadora en las columnas \textit{Integrantes} y \textit{Correos Integrantes},
respectivamente.

Este sistema fue diseñado para ser utilizado por la coordinación de titulación del DCC,
por lo que al agregar o editar comisiones toma en cuenta elementos como la cantidad de
comisiones a las que ha sido asignada cada profesor/a y que se puedan agregar varios
miembros a una comisión en un mismo formulario. Además, en el formulario se permite
buscar profesores por nombre y área de conocimiento, como se puede ver en la figura
\ref{subfig:ingsoft_buscar}.

\begin{figure}[ht]
    \begin{subfigure}[b]{0.4\textwidth}
        \centering
        \includegraphics[width=\linewidth]{imagenes/ingsoft_agregar_miembros.png}
        \caption{Formulario que permite agregar varios miembros a una comisión examinadora.}
        \label{subfig:ingsoft_agregar}
    \end{subfigure}
    \hfill
    \begin{subfigure}[b]{0.4\textwidth}
        \centering
        \includegraphics[width=\linewidth]{imagenes/ingsoft_buscar_miembros.png}
        \caption{Selector que permite buscar profesores por nombre y área de conocimiento.}
        \label{subfig:ingsoft_buscar}
    \end{subfigure}
    \caption{Formulario de asignación de comisiones examinadoras en el Sistema de Asignación de Comisiones.}
    \label{fig:ingsoft}
\end{figure}

Como se mencionó anteriormente, el Sistema de Asignación de Comisiones es un sistema
piloto, por lo que no tiene todas las características deseadas. En primer lugar, faltan
validaciones al momento de asignar miembros a una comisión examinadora, haciendo posible
asignar a los profesores guía y coguía como integrantes de la misma comisión examinadora,
quedando registrados dos veces en la comisión. En segundo lugar, no se encuentra bien
integrado con el sistema de titulación, sobre el cual fue desarrollado. Esto se debe a
que el sistema permite asignar comisiones examinadoras a memoristas y exportarlas, pero
las comisiones asignadas no son visibles en otros módulos del sistema de titulación, en
los que sería deseable verlas, como por ejemplo las fichas de los estudiantes. En tercer
lugar, el sistema no se encuentra directamente integrado con el SSM.
\chapter{Diseño}
En este capítulo se abordarán la arquitectura, el modelo de datos y el diseño de
interfaces del Sistema de Titulación. En la sección \ref{sec:arquitectura} se describirá
la arquitectura del Sistema de Titulación y cómo se relaciona con otros sistemas. Luego,
en la sección \ref{sec:modelo_datos} se describirá el modelo de datos del sistema de
titulación y los cambios por los que este pasó. Finalmente, en la sección
\ref{sec:mockups} se describirá el diseño de interfaces del Sistema de Titulación.

\section{Arquitectura}\label{sec:arquitectura}
A continuación se describirá la arquitectura del sistema de titulación. Primero se
describirá la arquitectura interna del sistema, es decir, los módulos que componen el
sistema. Luego, se describirá la arquitectura externa del sistema, es decir, cómo se
relaciona con otros sistemas.

El sistema de titulación es un sistema web hecho con el framework Django y utiliza una
base de datos PostgreSQL. Internamente se encuentra formado por varios módulos como se
puede ver en la figura \ref{fig:arquitectura_interna}, cada uno sigue la arquitectura
Modelo-Vista-Controlador (MVC), puesto que el framework Django ya implementa esta
arquitectura. Los módulos son los siguientes:
\begin{multicols}{2}
    \begin{itemize}
        \item Core
        \item Módulo de servicios
        \item Módulo de comisiones examinadoras
        \item Módulo de titulación
        \item Módulo de trabajo de título
        \item Módulo de docencia
        \item Módulo de evaluación
        \item Módulo de departamento
        \item Módulo de investigación
    \end{itemize}
\end{multicols}

El módulo Core contiene configuración y funcionalidades utilizada por todo el sistema. El
módulo de servicios se encarga de las notificaciones. El módulo de docencia se encarga de
gestionar los periodos académicos y los cursos. El módulo de investigación se encarga de
las áreas de investigación. El módulo de departamento se encarga de gestionar los
estudiantes, funcionarios y los roles de estos. El módulo de titulación contiene la
lógica relacionada con el ramo Introducción al Trabajo de Título, como los temas, las
solicitudes de temas y las memorias durante ese ramo. El módulo de trabajo de título
contiene la lógica relacionada con el ramo Trabajo de Título. Por último, el módulo de
comisiones examinadoras contiene la lógica relacionada con las asignación y exportación
de comisiones examinadoras.

\begin{figure}[ht]
    \centering
    \includegraphics[width=0.6\linewidth]{imagenes/arquitectura/arquitectura_interna_titulacion.png}
    \caption{Arquitectura interna del sistema de titulación.}
    \label{fig:arquitectura_interna}
\end{figure}

El sistema de titulación se relaciona con otros sistemas, ya sea consumiendo sus APIs o
exportando datos, como se puede ver en la figura \ref{fig:arquitectura_externa}. Específicamente,
el sistema de titulación se relaciona con el Sistema de Seguimiento de Memorias (SSM), la
API de UCampus llamada Mufasa y el Portal DCC.

\begin{figure}[ht]
    \centering
    \includegraphics[width=0.5\linewidth]{imagenes/arquitectura/arquitectura_macro_titulacion.png}
    \caption{Arquitectura externa del sistema de titulación.}
    \label{fig:arquitectura_externa}
\end{figure}

El sistema depende del Portal DCC para el inicio de sesión de los usuarios y la obtención de
los datos de estos, como roles y permisos. Por otro lado, el sistema utiliza la API de
UCampus para obtener datos de los periodos académicos, los ramos relacionados con el
proceso de titulación, sus respectivas secciones y los estudiantes inscritos en estas
secciones. Finalmente, el sistema exporta datos sobre las memorias y las comisiones
examinadoras asignadas al Sistema de Seguimiento de Memorias (SSM). Esta exportación
se realiza a través de la API de SSM.

\section{Modelo de Datos}\label{sec:modelo_datos}
En esta sección se describirá el modelo de datos del sistema de titulación y cómo
evolucionó durante este trabajo de título. Primero se describirá el modelo de datos en su
estado inicial en la subsección \ref{sec:ta:modelo_datos_inicial}, es decir, como estaba
antes de realizar este trabajo de título. Luego, en la subsección \ref{sec:ta:cambios_iniciales_modelo_datos}
se describirán los cambios que se planificaron inicialmente. Por último, se describirán los cambios
que surgieron durante el desarrollo de esta memoria.

\subsection{Modelo de Datos Inicial}\label{sec:ta:modelo_datos_inicial}
El repositorio del sistema de titulación tiene varios módulos y cada uno tiene sus
propias entidades que pueden relacionarse con las de otros módulos. No todas las
entidades son relevantes para la asignación de comisiones examinadoras, por lo
que solo se describirán las entidades que se relacionan con el módulo de comisiones
examinadoras y el módulo de titulación. Es importante mencionar que el modelo de datos
que se describirá a continuación es el modelo en su estado previo a este trabajo de título.

El módulo principal es el módulo de titulación, que se puede ver en la figura
\ref{fig:modelo_titulacion}. Este módulo contiene las entidades Tema, Solicitud,
Propuesta y Documento. Estas representan el tema de un trabajo de título, una solicitud
de un estudiante a un tema, un trabajo de título de un estudiante en el contexto del
ramo de Introducción al Trabajo de Título y los documentos con los informes que deben
entregar los estudiantes, respectivamente.

\begin{figure}[ht!]
    \centering
    \includegraphics[width=0.6\linewidth]{imagenes/diagramas/titulacion.png}
    \caption{Diagrama del modelo de datos del módulo de titulación.}
    \label{fig:modelo_titulacion}
\end{figure}

El módulo de comisiones examinadoras contiene a las entidades Comision y
AlumnoCursandoMemoria, como muestra la figura \ref{fig:modelo_comisiones}. Comisión
representa una comisión examinadora de un trabajo de título. Contiene una relación con la
entidad Tema, que representa el tema del trabajo de título, y una relación con la entidad
Solicitud, que representa la solicitud del tema. Además, tiene una relación de n a n con la
entidad Evaluador del módulo de evaluación, que representa los integrantes de la comisión
examinadora.

\begin{figure}[ht!]
    \centering
    \includegraphics[width=0.7\linewidth]{imagenes/diagramas/comisiones.png}
    \caption{Diagrama del modelo de datos del módulo de comisiones examinadoras.}
    \label{fig:modelo_comisiones}
\end{figure}

AlumnoCursandoMemoria representa a un estudiante que está cursando Trabajo de título.
Contiene al estudiante a través de una relación con la entidad Persona del módulo kernel
y se asocia a una memoria a través de la solicitud del tema. Además, contiene el periodo
académico en el que el estudiante cursa el ramo Trabajo de título y tiene una relación con
la entidad Comision, que representa la comisión examinadora que se le asigna al estudiante.
Por último, tiene el estado de la defensa, que indica si la defensa ha sido aprobada, reprobada o si
la defensa aún no ha sido realizada. Los posibles estados son \textit{pendiente},
\textit{aprobado} y \textit{reprobado}.


El módulo de evaluación contiene a las entidades Evaluador, que corresponden a personas
que evalúan, como los integrantes de una comisión examinadora. El módulo de departamento
proporciona las entidades Funcionario y Estudiante. Funcionario representa a
funcionarios del departamento y en el caso de titulación, representa a académicos del
departamento que son guías o coguías de trabajos de título. Estudiantes representa a
estudiantes del DCC. Las entidades de ambos módulos se pueden ver en la figura
\ref{fig:modelo_departamento}.

\begin{figure}[ht]
    \centering
    \includegraphics[width=0.89\linewidth]{imagenes/diagramas/departamento_evaluacion.png}
    \caption{Diagrama del modelo de datos de los módulos departamento y evaluación.}
    \label{fig:modelo_departamento}
\end{figure}


El módulo de docencia proporciona los periodos académicos y el
módulo de investigación proporciona Area, que representa áreas de conocimiento dentro de
computación y permite asignar áreas de conocimiento tanto a los académicos como a los
temas de trabajo de título. Por último, el módulo kernel proporciona la entidad
Persona, que representa a cualquier usuario de la plataforma.

\begin{figure}[ht]
    \centering
    \includegraphics[width=\linewidth]{imagenes/diagramas/investigacion.png}
    \caption{Diagrama del modelo de datos de los módulos docencia, investigación y kernel.}
    \label{fig:modelo_docencia_investigacion_kernel}
\end{figure}


\subsection{Cambios Iniciales al Modelo de Datos}\label{sec:ta:cambios_iniciales_modelo_datos}

Tomando en cuenta el modelo de datos actual y las funcionalidades que se desea agregar a
la herramienta, el modelo de datos no sufrirá muchos cambios, ya que la mayoría de las
funcionalidades trabajan con datos que ya se encuentran en el modelo. La única
funcionalidad que requiere un cambio es la de publicar las comisiones examinadoras, puesto
que se necesita diferenciar entre comisiones que han sido publicadas y aquellas que no lo
han sido. Para lograr esto, se agregará la entidad ComisionBorrador que tendrá los mismos
atributos que Comision, pero que se diferenciará por el hecho de que no ha sido
publicada. De esta forma, la publicación de comisiones examinadoras por API se realizará con
los datos de la entidad Comision.

\subsection{Cambios al Modelo de Datos Durante el Desarrollo}\label{sec:ta:cambios_modelo_datos_desarrollo}

A continuación se detallará por cada módulo las modificaciones realizadas al modelo de datos que no fueron
planificadas inicialmente, y la razón por la que se realizaron.

En primer lugar, en el módulo de titulación se agregó el atributo tema a la entidad
Propuesta, que corresponde al tema de la memoria. Este se agregó, puesto que
anteriormente solo se podía acceder al tema de una memoria a través de la entidad
Solicitud. Esto no era práctico y tampoco hace sentido, ya que Propuesta
representa una memoria en el ramo de Introducción al Trabajo de Título y toda memoria
debe tener un tema. Esto se puede ver en la figura \ref{fig:modelo_titulacion_nuevo}.

\begin{figure}[ht]
    \centering
    \includegraphics[width=0.7\linewidth]{imagenes/diagramas/modelo_titulacion_nuevo.png}
    \caption{Diagrama del modelo de datos del módulo de titulación.}
    \label{fig:modelo_titulacion_nuevo}
\end{figure}

Luego, en el módulo de docencia, se agregó la entidad Curso, que representa una sección
de un ramo en un periodo académico específico, lo cual se ve reflejado en la figura
\ref{fig:modelo_docencia_nuevo}. Esta entidad tiene los atributos periodo, número de
sección, código del ramo, nombre del ramo, id del ramo e id del curso. Estos dos últimos
corresponden a identificadores dentro de la API de UCampus. Se agregó esta entidad para
identificar el curso al que pertenece una memoria y así poder exportar las comisiones
examinadoras asignadas al SSM que separa las memorias por curso.

\begin{figure}[ht!]
    \centering
    \includegraphics[width=0.32\linewidth]{imagenes/diagramas/modelo_docencia_nuevo.png}
    \caption{Diagrama del modelo de datos del módulo de docencia.}
    \label{fig:modelo_docencia_nuevo}
\end{figure}

En el módulo de comisiones examinadoras se eliminó el atributo solicitud de la entidad
Comision, puesto que no es necesario que desde una comisión se acceda a la solicitud al
tema de la memoria, ya que la solicitud solo es relevante al comienzo del ramo
Introducción al Trabajo de Título y las comisiones examinadoras aparecen recién en el
ramo Trabajo de Título. También se eliminó la entidad ComisionBorrador, puesto que no
es necesario tener una entidad para comisiones examinadoras que no han sido publicadas,
en su lugar, se agregó el atributo booleano publicada a la entidad Comision, que indica
si la comisión ha sido exportada al SSM. Además, se eliminó la entidad
AlumnoCursandoMemoria, ya que la entidad representaba a un estudiante que cursaba una
memoria en el ramo Trabajo de Título, lo cual no era consistente con el modelo de datos,
pues ya existe la entidad Estudiante y su equivalente en el ramo de Introducción al
Trabajo de Título, es decir Propuesta, representa a la memoria en el ramo de Introducción
al Trabajo de Título y no al estudiante que cursaba la memoria. La entidad fue reemplazada
por la entidad MemoriaEnF, que será descrita a continuación.

Por último, se creó el módulo de Trabajo de Título que contiene la entidad MemoriaEnF.
Esta entidad representa una memoria en el ramo de Trabajo de Título y tiene como atributos
al estudiante que realiza la memoria, la comisión examinadora, el tema de la memoria, el periodo
académico y curso en el que se realiza la memoria, el tipo de memoria, el estado de defensa y una
referencia a la entidad Propuesta, que corresponde a la misma memoria, pero en el ramo de
Introducción al Trabajo de Título. Tanto la entidad MemoriaEnF como la entidad Comision
se pueden ver en la figura \ref{fig:modelo_trabajo_titulo_nuevo}.

\begin{figure}[ht]
    \centering
    \includegraphics[width=0.9\linewidth]{imagenes/diagramas/modelo_trabajotitulo_comision_nuevo.png}
    \caption{Diagrama del modelo de datos de los módulos de comisiones examinadoras y trabajo de título.}
    \label{fig:modelo_trabajo_titulo_nuevo}
\end{figure}


\section{Diseño de Mockups}\label{sec:mockups}
En esta sección se presentan los mockups de las interfaces principales del módulo de
comisiones examinadoras.

\subsection{Interfaz principal del módulo de comisiones examinadoras}
\label{subsec:mockups:comisiones}
El primer diseño en ser realizado fue el de la interfaz principal del módulo de
asignación de comisiones examinadoras. La interfaz original se puede ver en la figura
\ref{fig:ingsoft_comisiones} y mientras que la propuesta se puede ver en la figura
\ref{fig:mockup_listado_comisiones}.

\begin{figure}[ht]
    \centering
    \includegraphics[width=0.77\linewidth]{imagenes/mockups/listado_comisiones.png}
    \caption{Mockup de la interfaz principal del módulo de comisiones examinadoras.}
    \label{fig:mockup_listado_comisiones}
\end{figure}

Ambas interfaces son similares, ya que la tabla mantiene la misma estructura. Comenzando
por la parte superior, se mantiene el selector de periodo académico y el botón de
\textit{Sincronizar}. El botón \textit{Descargar} es equivalente al botón
\textit{Exportar} de la interfaz original, solo cambia el texto y el ícono. Se agregó un
botón con ícono de gráfico que permite mostrar y esconder un gráfico con la cantidad de
comisiones examinadoras y memorias guíadas por cada profesor en el periodo académico
seleccionado, que se muestra en la figura \ref{fig:mockup_grafico_comisiones}, las columnas
están ordenadas de forma descendiente por la cantidad de carga de cada profesor. Como se
trata de varios profesores, se decidió que el gráfico no tenga etiquetas en el eje X,
pues no serían visibles y saturaría el gráfico. En su lugar, se muestra el nombre del profesor y
la cantidad de comisiones examinadoras y memorias guíadas al pasar el mouse por encima de
una barra del gráfico.

\begin{figure}[ht]
    \centering
    \includegraphics[width=0.8\linewidth]{imagenes/mockups/grafico_comisiones.png}
    \caption{Mockup del gráfico de comisiones examinadoras.}
    \label{fig:mockup_grafico_comisiones}
\end{figure}

También se agregó el botón \textit{Publicar} que permite publicar las comisiones
examinadoras en el sistema de titulación, permitiendo que las comisiones asignadas se
muestren en la ficha del estudiante y estén disponibles para exportar. Al presionar
\textit{Publicar}, aparece un modal con un mensaje de confirmación, indicando si es que
quedan memorias sin comisión asignada y cuántas son, en caso de que haya. Esto se puede
ver en la figura \ref{fig:mockup_confirmacion_subida_comisiones}.

\begin{figure}[ht]
    \centering
    \includegraphics[width=0.7\linewidth]{imagenes/mockups/confirmacion_subida_comisiones.png}
    \caption{Mockup de la confirmación de la publicación de comisiones examinadoras.}
    \label{fig:mockup_confirmacion_subida_comisiones}
\end{figure}

Más abajo, al costado del buscador, se agregó un filtro que permite ajustar si en la
tabla se muestran todas las memorias, solo las memorias con comisión incompleta o solo
las memorias con comisión completa. Este filtro se puede ver en la figura \ref{fig:mockup_filtro_comisiones}.

\begin{figure}[ht]
    \centering
    \includegraphics[width=0.8\linewidth]{imagenes/mockups/filtro_comisiones_zoom.png}
    \caption{Mockup del filtro de la interfaz principal del módulo de comisiones examinadoras.}
    \label{fig:mockup_filtro_comisiones}
\end{figure}

Por último, en la tabla se decidió que en los botones de acción, es decir,
\textit{Agregar} y \textit{Editar}, se mostrarán solo los íconos de + y lápiz,
respectivamente, porque se consideró que sin el texto se mantendría la claridad de la
acción que realizan y se disminuiría la densidad de texto en la tabla.


\subsection{Ficha de un estudiante con comisión asignada}
Después, se diseñó la interfaz de la ficha de un estudiante desde la vista de un
estudiante con la comisión examinadora que le ha sido asignada, que se puede ver en la
figura \ref{fig:mockup_ficha_estudiante}. No se cambió nada de la interfaz original,
solo se agregó la sección de la comisión examinadora. Esta sección se ubica bajo la
sección de Tema en una tabla con el mismo formato visual que las otras tablas de la ficha
del estudiante. Cada fila corresponde a un integrante de la comisión examinadora,
indicando el rol dentro de la comisión, una imagen del integrante, su nombre y correo
electrónico.

\begin{figure}[ht]
    \centering
    \includegraphics[width=0.8\linewidth]{imagenes/mockups/ficha_estudiante.png}
    \caption{Mockup de la ficha de un estudiante desde la vista de un estudiante
        con la comisión examinadora asignada.}
    \label{fig:mockup_ficha_estudiante}
\end{figure}

\chapter{Implementación}
En este capítulo se describirá la implementación realizada durante la memoria.
Esta se separa puede separar en tres partes, la primera es la implementación relacionada
con los estudiantes del ramo Trabajo de Título, también llamado F. La segunda parte
corresponde a la asignación de comisiones examinadoras. La tercera parte corresponde a la
exportación de comisiones examinadoras.

\section{Estudiantes de Trabajo de Título}

\subsection{Módulo de Trabajo de Título}
Para implementar funcionalidades relacionadas con el ramo Trabajo de Título y
separarlas de las funcionalidades del ramo Introducción al Trabajo de Título, se creó el
módulo de Trabajo de Título. Esta separación también evita que la barra de navegación se
llene de pestañas. Para acceder al módulo, se creó un selector en la barra de navegación,
que se ubica en la parte superior de la página, como se puede ver en la figura
\ref{fig:implementacion_selector}. Como por el momento las únicas funcionalidades asociadas
al módulo de Trabajo de Título son ver los estudiantes de Trabajo de Título y la asignación
de comisiones examinadoras, y ambas requieren permisos de coordinación de titulación, el
selector solo es visible para coordinadores, que está manejado por la variable booleana
\verb|show_module_selector|, como se muestra en la línea 11 del código \ref{lst:barra_navegacion}.
Al seleccionar el módulo Introducción al Trabajo de Título, el usuario es redirigido a la
raíz de la aplicación. Mientras que al seleccionar el módulo Trabajo de Título, el usuario
es redirigido a la vista principal de Trabajo de Título, que es el listado de estudiantes de
Trabajo de Título.

Como anteriormente todo el sistema estaba dedicado al ramo Introducción al Trabajo de
Título, el título de la barra de navegación era el nombre del módulo. Por lo tanto, se
creó el módulo Trabajo de Título para poder separarlo del módulo de Introducción al
Trabajo de Título. Además, el título de la barra de navegación se cambió a un título
configurable mediante la variable \verb|titulo| y si esta variable no se especifica, se
muestra el nombre del módulo. Se decidió dejar este caso, ya que el header es el mismo para
todos los módulos y el resto de módulos no requieren un título personalizado.


\begin{figure}[ht]
  \begin{subfigure}[b]{0.47\textwidth}
    \centering
    \includegraphics[width=\linewidth]{imagenes/implementaciones/selector_seccion_cerrado.png}
    \caption{Captura de pantalla del selector cerrado en la barra de navegación en el módulo de Introducción al Trabajo de Título.}
    \label{fig:implementacion_selector_cerrado}
  \end{subfigure}
  \hfill
  \begin{subfigure}[b]{0.47\textwidth}
    \centering
    \includegraphics[width=\linewidth]{imagenes/implementaciones/selector_seccion_abierto.png}
    \caption{Captura de pantalla del selector abierto en la barra de navegación en el módulo de Introducción al Trabajo de Título.}
    \label{fig:implementacion_selector_abierto}
  \end{subfigure}
  \hfill
  \begin{subfigure}[b]{0.47\textwidth}
    \centering
    \includegraphics[width=\linewidth]{imagenes/implementaciones/selector_seccion_trabajo_titulo.png}
    \caption{Captura de pantalla del selector cerrado en la barra de navegación en el módulo de Trabajo de Título.}
    \label{fig:implementacion_selector_trabajo_titulo}
  \end{subfigure}
  \hfill
  \begin{subfigure}[b]{0.47\textwidth}
    \centering
    \includegraphics[width=\linewidth]{imagenes/implementaciones/selector_seccion_trabajo_titulo_abierto.png}
    \caption{Captura de pantalla del selector abierto en la barra de navegación en el módulo de Trabajo de Título.}
    \label{fig:implementacion_selector_trabajo_titulo_abierto}
  \end{subfigure}
  \caption{Capturas de pantalla del selector en la barra de navegación.}
  \label{fig:implementacion_selector}
\end{figure}

\begin{lstlisting}[language=HTML, caption={Código de la parte central de la barra de navegación}, label={lst:barra_navegacion}]
    <div class="header">
      <a href="" class="logo"><span class="dcc">dc<span>c</span></span></a>
      <div class="d-flex flex-column flex-grow-1 align-items-center pt-2">
        <h1 class="modulo mb-0">
          
            {{ titulo }}
          
            {{ modulo.nombre }}
          
        </h1>
        
          <div class="dropdown-center">
            <button class="btn btn-dark dropdown-toggle" type="button" data-bs-toggle="dropdown">
              {{ modulo.nombre }}
            </button>
            <ul class="dropdown-menu">
              
                <li>
                  <a class="dropdown-item  disabled "
                     href=" #  {{ mod.url }} ">
                    {{ mod.nombre }}
                  </a>
                </li>
              
            </ul>
          </div>
        
      </div>
    </div>
\end{lstlisting}

\subsection{Listado de estudiantes de Trabajo de Título}

La asignación de comisiones examinadoras a los estudiantes de Trabajo de Título, requiere
tener un registro de los estudiantes que están cursando Trabajo de Título junto con sus
respectivos temas de memoria. Por lo tanto, dentro del módulo de Trabajo de Título se
creó una vista que lista los estudiantes que están cursando Trabajo de Título, que se
puede ver en la figura \ref{fig:listado_estudiantes_f}.

En esta vista se muestra una tabla que por cada estudiante muestra su nombre, el título
de su memoria, su guía, su coguía si tiene y un botón que redirige a su ficha. Sobre la
tabla al costado derecho se encuentra una barra de búsqueda que permite buscar por nombre
de estudiante, título de memoria, nombre de guía y nombre de coguía. Sobre esta barra, se
encuentra un filtro que permite elegir el semestre académico de los estudiantes que se
mostrarán en la tabla. Por defecto se mostrarán los estudiantes del semestre académico
activo. Por último, al costado derecho de este filtro está el botón de
\textit{Sincronizar}, que permite actualizar la tabla con los estudiantes.

\begin{figure}[ht]
  \centering
  \includegraphics[width=\linewidth]{imagenes/implementaciones/listado_estudiantes_f.png}
  \caption{Captura de pantalla de la vista del listado de estudiantes de Trabajo de Título.}
  \label{fig:listado_estudiantes_f}
\end{figure}

Para la actualización del registro se tomó en cuenta que los estudiantes deben estar
cursando alguna sección de los ramos de Trabajo de Título, CC6909 y CC6919, por lo
tanto se puede obtener los nombres de los estudiantes desde la API de UCampus. Además,
los estudiantes deben haber aprobado Introducción al Trabajo de Título, por lo que se
asume que los temas de los estudiantes ya se encuentran registrados en el sistema de
titulación y que también hay una instancia del modelo Propuesta para cada estudiante
asociada a su tema y con estado aprobado, haciendo posible asociar a los estudiantes que
cursan Trabajo de Título con sus temas.

Dada la asunción anterior, para actualizar el registro de estudiantes de Trabajo de
Título dentro del sistema de titulación se creó un comando de administración
\verb|upd_estudiantes_f|, que es llamado por el botón \textit{Sincronizar}.
Este obtiene desde la API de UCampus todas las secciones de
los ramos de Trabajo de Título y para después obtener los integrantes de cada sección.
Luego, por cada integrante se busca la instancia del modelo Propuesta con estado
aprobado más reciente asociada al estudiante, pues puede haber varias instancias
asociadas al mismo estudiante. Si se encuentra una instancia, se obtiene el tema asociado
y se crea una instancia del modelo MemoriaEnF con el estudiante y su tema. En caso
contrario, se registra en el log que para ese estudiante no se encontró una instancia,
se agrega al estudiante a la lista de estudiantes sin una instancia y se continua con el
siguiente integrante. Al finalizar el proceso se muestra un mensaje indicando cuántos
estudiantes no tuvieron una instancia, como se ve en la figura \ref{fig:estudiante_sin_tema}.
Una vez que termina la ejecución del comando, se recarga la página y se muestra la lista
actualizada de estudiantes de Trabajo de Título.

\begin{figure}[ht]
  \centering
  \includegraphics[width=0.6\linewidth]{imagenes/implementaciones/estudiante_sin_tema.png}
  \caption{Captura de pantalla de la notificación de que un estudiante no tiene un tema registrado.}
  \label{fig:estudiante_sin_tema}
\end{figure}

Como se mencionó en la descripción del comando, puede suceder que haya estudiantes que no
tengan una instancia de Tema y Propuesta con estado Aprobado. Esto sucede por lo que la
asunción puede no ser cierta para estudiantes que hayan solicitado la vía rápida, pues
actualmente ese proceso no se encuentra implementado en el sistema de titulación y se
realiza directamente con la coordinación de titulación. El caso de la vía rápida no
se manejará en esta memoria, ya que al momento del desarrollo de este trabajo de título,
había otro tema de memoría sobre la incorporación de un módulo de vía rápida. Mientras
que tanto, basta con crear una instancia de Tema y Propuesta con estado Aprobado para
cada estudiante de vía rápida mediante el administrador de Django o por consola para que
puedan ser incluidos en la lista de estudiantes de Trabajo de Título.

\subsection{Vista de \textit{Mis Memoristas} para académicos}
Habiendo incorporado a los estudiantes de Trabajo de Título al sistema de titulación,
se extendió la vista de \textit{Mis Memoristas} que es visible para académicos, para que
pudieran ver una lista de sus memoristas en Trabajo de Título. Para ello se agregó un
selector de ramo de titulación que permite elegir entre Introducción al Trabajo de Título
y Trabajo de Título. Este se encuentra bajo la barra de navegación de la vista, a la
izquierda del filtro de periodo académico. En la figura
\ref{subfig:implementacion_mis_memoristas_e} se muestra el selector en la vista con memoristas
de Introducción al Trabajo de Título, mientras que en la figura
\ref{subfig:implementacion_mis_memoristas_f} se muestra el selector en la vista con memoristas
de Trabajo de Título.

Al seleccionar Trabajo de Título, se muestra la lista de los estudiantes de Trabajo de Título
que son memoristas del usuario. Por cada estudiante se muestra su nombre, su título de memoria,
su guía, su coguía si tiene y un botón a la ficha del estudiante. El resto de la vista se mantiene
igual que la vista para Introducción al Trabajo de Título.

\begin{figure}[ht]
  \centering
  \begin{subfigure}[t]{\textwidth}
    \centering
    \includegraphics[width=\linewidth]{imagenes/implementaciones/mis_memoristas_e.png}
    \caption{Captura de pantalla de la vista de \textit{Mis Memoristas} para Introducción al Trabajo de Título.}
    \label{subfig:implementacion_mis_memoristas_e}
  \end{subfigure}
  \begin{subfigure}[t]{\textwidth}
    \centering
    \includegraphics[width=\linewidth]{imagenes/implementaciones/mis_memoristas_f.png}
    \caption{Captura de pantalla de la vista de \textit{Mis Memoristas} para Trabajo de Título.}
    \label{subfig:implementacion_mis_memoristas_f}
  \end{subfigure}
  \caption{Vista de \textit{Mis Memoristas} para académicos con selector de ramo de titulación.}
  \label{fig:implementacion_mis_memoristas}
\end{figure}

% \begin{lstlisting}[language=HTML]
% <div class="form-responsive-container">
%   <select class="filtro form-select"
%           onchange="window.location = this.options[this.selectedIndex].value;">
%     
%       <option value="{{ m.mis_memoristas_url }}" selected>{{ m.nombre }}</option>
%     
%   </select>
% </div>
% \end{lstlisting}



\section{Asignación de Comisiones}
En esta sección se describe la implementación de funcionalidades relacionadas con la
asignación de comisiones examinadoras. Primero se mencionarán las funcionalidades que se
implementaron para la evaluación de la herramienta, y luego se describirán las
modificaciones que se surgieron a partir de la retroalimentación recibida en la
evaluación.

El sistema piloto desarrollado por el equipo de ingeniería de software II ya contaba con
la funcionalidad de asignación de comisiones examinadoras, tal como se menciona en la
sección \ref{sec:sa:asignacion}. Sin embargo, esta funcionaba con el modelo
AlumnoCursandoMemoria, por lo que se adaptó para que funcione con el nuevo modelo
MemoriaEnF, manteniendo la tabla de comisiones examinadoras que muestra una lista de
memorias con sus respectivos estudiantes, temas, guías, coguías e integrantes de la comisión.
Al final de cada fila se encuentra un botón que permite asignar evaluadores a la comisión. La figura
\ref{fig:comisiones_asignacion} muestra la lista de memorias con sus comisiones examinadoras .

\begin{figure}[ht]
  \centering
  \includegraphics[width=0.8\linewidth]{imagenes/implementaciones/lista_comisiones.png}
  \caption{Captura de pantalla de la lista de memorias con sus comisiones examinadoras.}
  \label{fig:comisiones_asignacion}
\end{figure}

\subsection{Filtro de Comisiones Examinadoras}
Una vez adaptada la implementación al nuevo modelo, se le agregaron funcionalidades.
En primer lugar, se agregó un filtro de comisiones examinadoras en la interfaz
principal del módulo de comisiones examinadoras. Este filtro permite configurar los
trabajos de título que se muestran dependiendo de la cantidad de evaluadores asignados a
la comisión examinadora. Si se selecciona ``Todas las memorias'', se muestran todos los
trabajos de título, si se selecciona ``Con comisión completa'', se muestran solo aquellas
que tengan dos o más evaluadores asignados, y si se selecciona ``Con comisión
incompleta'', se muestran aquellas que tengan un evaluador o ninguno. En la figura
\ref{fig:comisiones_filtradas} se muestra el filtro configurado para mostrar solo las
memorias con comisión completa. Internamente, el filtro refresca la página y agrega un
query parameter a la URL que indica el filtro seleccionado. Desde el backend se filtran
las comisiones según el query parameter, como se muestra en el código \ref{lst:filtro_comisiones}.

\begin{figure}[ht]
  \centering
  \includegraphics[width=0.8\linewidth]{imagenes/implementaciones/comisiones_filtradas.png}
  \caption{Captura de pantalla de la lista de comisiones examinadoras filtrada por comisiones completas.}
  \label{fig:comisiones_filtradas}
\end{figure}

\begin{lstlisting}[language=Python, caption={Filtrado de comisiones examinadoras.}, label={lst:filtro_comisiones}]
  memorias = MemoriaEnF.objects.filter(periodo=periodo).select_related(
    "tema__guia__persona",
    "tema__coguia__persona",
    "estudiante__persona",
  ).prefetch_related(
    "comision__evaluadores"
  ).annotate(
    Count("comision__evaluadores")
  )
  # Filtrado por comision completa, incompleta o todas
  if comision_filter == "completas":
    memorias_filtradas = memorias.filter(comision__evaluadores__count__gte=2)
  elif comision_filter == "incompletas":
    memorias_filtradas = memorias.filter(comision__evaluadores__count__lt=2)
  else:
    memorias_filtradas = memorias

  # Ordenar: primero con tema (True > False)
  memorias_filtradas = sorted(
    memorias_filtradas,
    key= lambda memoria: (
      memoria.tema is None, # False (tienen tema) va antes
      memoria.estudiante.persona.apellido1.lower() if memoria.estudiante.persona.apellido1 else "",
      memoria.estudiante.persona.nombre1.lower() if memoria.estudiante.persona.nombre1 else "",
    )
  )
  context["memorias"] = memorias_filtradas
\end{lstlisting}


\subsection{Gráfico de Carga de Evaluadores}
El mayor cambio visual en la interfaz del módulo de comisiones proviene de la
incorporación del gráfico que muestra la carga de evaluadores. Este se ubica sobre la
lista de comisiones examinadoras, como se muestra en la figura
\ref{fig:grafico_carga_evaluadores}. Cada columna representa la carga de un evaluador,
que está dividida entre la carga proveniente de las memorias guíadas o coguiadas y la carga
proveniente de las comisiones examinadoras que integra. Es importante destacar que integrar
una comisión examinadora equivale a un punto de carga, mientras que guiar o coguir una memoria
equivale a dos puntos de carga. Esta ponderación fue recomendada por la coordinadora de titulación.
Al posicionar el cursor sobre una columna, se muestra un tooltip que indica la cantidad de comisiones
y la cantidad de memorias.

El gráfico está ordenado de mayor a menor función de la carga total de cada evaluador, es
decir, por la siguiente fórmula: $carga\_total = 2*memorias\_guiadas + comisiones\_integradas$.
Este orden ayuda a que sea fácil identificar a los evaluadores con mayor y menor carga.
Además, el gráfico se actualiza cada vez que se asignan integrantes a una comisión examinadora y
no se ve afectado por el filtro de comisiones. Por último, el gráfico puede ocultarse y mostrarse
mediante un botón que se encuentra bajo la barra de navegación, entre el filtro de comisiones
y el botón de sincronización.

\begin{figure}[ht]
  \centering
  \includegraphics[width=\linewidth]{imagenes/implementaciones/grafico_evaluadores.png}
  \caption{Captura de pantalla del gráfico de carga de evaluadores.}
  \label{fig:grafico_carga_evaluadores}
\end{figure}


\subsection{Modal de asignación de comisiones}
Al presionar el botón para asignar integrantes a una comisión examinadora, se abre un modal
para realizar la asignación. Este modal también se encontraba en la versión piloto del sistema,
pero al igual que la tabla de comisiones examinadoras, se adaptó para que use el modelo
MemoriaEnF en vez de AlumnoCursandoMemoria. Además, en el modal se agregó el nombre del estudiante.
En la figura \ref{fig:modal_asignacion_comisiones} se muestra el modal con los cambios realizados.

\begin{figure}[ht!]
  \centering
  \includegraphics[width=0.5\linewidth]{imagenes/implementaciones/modal.png}
  \caption{Captura de pantalla del modal de asignación de comisiones.}
  \label{fig:modal_asignacion_comisiones}
\end{figure}

Un cambio importante es que se agregaron validaciones en la asignación de comisiones. En
particular, se valida que al guardar una comisión examinadora, esta tenga al menos dos
integrantes, que ninguno de los integrantes sea guía o coguía de la memoria y que al
menos uno de los integrantes tenga jerarquía de Académico de Jornada Completa (AJC) o
Académico de Jornada Parcial (AJP). Si se intenta guardar una comisión examinadora que
no cumpla con estas validaciones, se muestra un mensaje de error. En la figura
En el código \ref{lst:validacion_comision} se muestra la implementación de las
validaciones, mientras que en la figura \ref{fig:validaciones_comision} se muestran los
mensajes de error por validación.

\begin{lstlisting}[language=Python, caption={Validación de comisión examinadora.}, label={lst:validacion_comision}, escapechar=|]
# Validaci|\'o|n de comisi|\'o|n
if len(evaluadores) < 2:
    return JsonResponse({"success": False, "message": "Debe agregar al menos dos evaluadores."})

rut_guia = comision.get_guia.persona.rut if comision.memoria.tema.guia else None
rut_coguia = comision.get_coguia.persona.rut if comision.memoria.tema.coguia else None

has_non_pex = False
for evaluador in evaluadores:
  if evaluador.persona.rut in (rut_guia, rut_coguia):
    return JsonResponse({"success": False, "message": "No puede seleccionar a los gu|\'i|as como evaluadores."})

  if evaluador.tipo.nombre != "PEX":
    has_non_pex = True

if not has_non_pex:
  return JsonResponse({"success": False, "message": "Debe elegir al menos un integrante AJC o AJP."})

comision.evaluadores.add(*evaluadores)
\end{lstlisting}

\newpage

\begin{figure}[ht!]
  \centering
  \begin{subfigure}[b]{0.45\textwidth}
    \centering
    \includegraphics[width=\linewidth]{imagenes/implementaciones/validacion_minimo_2.png}
    \caption{Mensaje de error indicando que se debe agregar al menos dos evaluadores.}
    \label{subfig:validacion_minimo_2}
  \end{subfigure}
  \hfill
  \begin{subfigure}[b]{0.45\textwidth}
    \centering
    \includegraphics[width=\linewidth]{imagenes/implementaciones/validacion_guia_evaluador.png}
    \caption{Mensaje de error indicando que un integrante de la comisión es guía o coguía.}
    \label{subfig:validacion_guia_evaluador}
  \end{subfigure}
  \begin{subfigure}[b]{0.45\textwidth}
    \centering
    \vspace{0.5cm}
    \includegraphics[width=\linewidth]{imagenes/implementaciones/validacion_ajc_ajp.png}
    \caption{Mensaje de error indicando que al menos un integrante debe ser ajc o ajp.}
    \label{subfig:validacion_ajc_ajp}
  \end{subfigure}
  \caption{Mensajes de error por validaciones de comisión examinadora.}
  \label{fig:validaciones_comision}
\end{figure}


\subsection{Cambios por Retroalimentación}\label{subsec:cambios_retroalimentacion}
Luego de realizar la evaluación de la herramienta (capítulo \ref{cap:validacion}),
se recibió una retroalimentación que indicaba que se debían hacer ciertos cambios en el
sistema. El punto de mayor importancia en la retroalimentación era que al presionar el
botón para asignar la comisión examinadora de una memoria demoraba mucho tiempo en aparecer
el modal, pues podía demorar hasta 15 segundos. Esta cantidad de tiempo es inaceptable para
la experiencia del usuario, en especial porque es un botón que se debe presionar por cada memoria.
Por lo tanto, se revisó el código del modal para ver posibles problemas y optimizaciones.

El problema se encontraba en el controlador del modal, que es llamado cada vez que se
presiona el botón para abrir el modal y responde con el HTML del modal. Que se llame
cada vez no es un problema, pero el trabajo que realizaba el controlador en cada llamada
era innecesariamente excesivo, ya que por cada llamada se obtenían todos los evaluadores
habílitados y a cada uno se le generaba un string con sus áreas de conocimiento con sus
respectivas subareas. Este proceso se realizaba para poblar el selector de evaluadores en
el modal. Para optimizar el proceso de poblar el selector de evaluadores, se tomaron dos
medidas: Precalcular el string con las áreas de conocimiento y obtener una única vez los
evaluadores.

Para realizar la primera medida, se agregó el atributo \verb|areas_string| al modelo Evaluador,
que almacena el string con las áreas de conocimiento de un evaluador. Además, se agregó
un método para calcular el string con las áreas de conocimiento de un evaluador y se
añadió una señal que se llama al método cada vez que haya un cambio en las áreas
del evaluador. Esto permite que el string solo se calcule cuando es necesario y no cada
vez que se llama el controlador del modal. En los códigos \ref{lst:senal_evaluador} y
\ref{lst:metodo_generate_areas_string} se muestra el código de la señal y el método
respectivamente.

\begin{lstlisting}[language=Python, caption={Código de la señal que actualiza el string con las áreas de conocimiento de un evaluador}, label={lst:senal_evaluador}, escapechar=|]
@receiver(m2m_changed, sender=Evaluador.areas.through)
def update_areas_string(sender, instance, action, **kwargs):
    """
    Actualiza el campo areas_string de un evaluador cada vez que cambian sus |\'a|reas.
    """
    if action in {"post_add", "post_remove", "post_clear"}:
        instance.generate_areas_string()
\end{lstlisting}

\begin{lstlisting}[language=Python, caption={Código del método para generar el string con las áreas de conocimiento de un evaluador}, label={lst:metodo_generate_areas_string}, escapechar=|]
    def generate_areas_string(self):
        """
        Genera un string con las |\'a|reas del evaluador y lo guarda en el atributo areas_string.
        El string tiene el siguiente formato: Padre - Hijo - Hijo - Hijo, Padre, Padre - Hijo
        """
        areas_by_parent = {}
        for area in self.areas.all():
            # Obtenemos el |\'a|rea padre (puede ser None)
            parent = area.padre
            parent_name = parent.nombre if parent else None
            child_name = area.nombre

            if parent_name:
                areas_by_parent.setdefault(parent_name, set()).add(child_name)
            else:
                if child_name not in areas_by_parent:
                    areas_by_parent[child_name] = set()

            # Construimos el string final
        result = []
        for parent, children in sorted(areas_by_parent.items()):
            if children:
                hijos_ordenados = sorted(children)
                result.append(f"{parent} - " + " - ".join(hijos_ordenados))
            else:
                result.append(parent)

        # Unimos todas las areas padre separadas por comas
        # Padre - Hijo - Hijo - Hijo, Padre, Padre - Hijo
        self.areas_string = ", ".join(result)
        self.save()
\end{lstlisting}

Para la segunda medida, se dejó de obtener a los evaluadores en el controlador del modal
y pasaron a obtenerse una vez al cargar la página del módulo de comisiones. Esto no significa
un mayor costo, ya que los evaluadores ya eran obtenidos al cargar la página, dado que el
gráfico de carga de evaluadores ya los obtenía. Estos evaluadores se almacenan en un
arreglo de objeto, almacenando su id, nombre, áreas de conocimiento, memorias guíadas,
comisiones integradas y jerarquía, para que puedan ser utilizados en el modal y el gráfico..

Un segundo punto a mejorar indicado en la retroalimentación fue al guardar una comisión,
se refrescaba la página, lo cual hacía que el usuario tuviera que bajar nuevamente hasta
la memoria que estaba viendo, lo cual es trabajo innecesario. Para resolver este problema,
se evitó refrescar la página y que al guardar una comisión el listado de integrantes se actualice
mediante JavaScript, se actualice el contenido del arreglo que se utiliza para poblar el listado de integrantes
y se actualice el gráfico de carga de integrantes.


\section{Exportación de Comisiones}
Para exportar las comisiones examinadoras, el sistema ofrece dos opciones: descargar
un archivo CSV y enviarlas a través de una petición POST la API del SSM.

\subsection{Archivo CSV}
El piloto desarrollado por el equipo de ingeniería de software II ya contaba con la
funcionalidad de exportar las comisiones examinadoras a un archivo CSV. Sin embargo,
el formato del archivo no seguía el formato requerido por el SSM, como se mencionó en las
secciones \ref{sec:sa:ssm} y \ref{sec:sa:asignacion}. Por lo tanto, se cambiaron las
columnas del archivo para que cumplan con los requisitos del SSM, se cambió el separador
al carácter coma y se agregaron las columnas que faltaban. Por último, se agregó la
condición de que solo las memorias con comisión completa sean exportadas


\subsection{Exportación directa al SSM}
Se implementó una exportación directa al SSM a través de una petición POST al SSM.
Esta implementación consta de dos partes. La primera es hacer que el SSM sea capaz de
recibir una petición desde otro sistema y procesarla. Mientras que la segunda es que el
sistema de titulación envíe una petición POST al SSM con los datos de las comisiones examinadoras.
Esta funcionalidad fue implementada luego de la evaluación por falta de tiempo.

\subsubsection{Adaptación del SSM}\label{subsubsec:adaptacion_ssm}
Para adaptar el SSM a recibir una petición POST desde otro sistema, se creó una view
de Django en el endpoint \texttt{/api/comisiones/} que recibe la petición POST con un
JSON en el cuerpo de la petición. Debido a que en el SSM las memorias se separan por
curso, es decir, por ramo y por sección, se decidió que en el JSON recibido, las
comisiones deben estar separadas por cursos. Además, es necesario que este contenga la
información necesaria para crear los cursos en la base de datos, en caso de que alguno
no exista. Esta corresponde al código del ramo, el número de la sección, el año y el
semestre, siendo ``O'' para el semestre de otoño y ``P'' para el semestre de primavera,
ya que así lo tiene definido el SSM. Por último, por cada comisión se debe tener el tema
de la memoria,nombre y correo del estudiante, del guía y del coguía, además de la lista
de integrantes. El campo coguía puede ser nulo si no hay co-guía. Un ejemplo de JSON
recibido con un solo curso y una sola comisión se puede ver en el código
\ref{lst:ssm_json}.
\newpage
\begin{lstlisting}[label=lst:ssm_json, caption={Ejemplo de JSON recibido con un solo curso y una sola comisión.}]
{
  "cursos": [{
    "codigo": "CC6919",
    "seccion": 1,
    "semestre": "P",
    "anno": 2025,
    "memorias": [{
      "estudiante": {
        "nombre": "Vicente Olivares",
        "correo": "vicente.olivares@ug.uchile.cl"
      },
      "tema": "Tema de ejemplo",
      "guia": {
        "nombre": "Nombre de ejemplo",
        "correo": "guia@example.com"
      },
      "coguia": null,
      "integrantes": [{
        "nombre": "Integrante de ejemplo 1",
        "correo": "integrante1@example.com"
      }, {
        "nombre": "Integrante de ejemplo 2",
        "correo": "integrante2@example.com"
      }]
    }]
  }]
}
\end{lstlisting}

Al recibir el JSON, primero se valida el formato de este. Se verifica que tenga la clave
\texttt{cursos} y que su valor no sea nulo, y se valida que las memorias de cada curso
contengan los campos \texttt{estudiante}, \texttt{tema}, \texttt{guia} y
\texttt{integrantes}. En segundo lugar, se verifica que el contenido no tenga
inconsistencias. Para esto se corrobora que no haya dos comisiones con el mismo
estudiante y que dentro de cada comisión, el guía, coguía e integrantes sean diferentes.
Si cualquiera de estas validaciones falla, se retorna un error 400 BadRequest con un mensaje
indicando qué validación falló.

En tercer lugar, se guardan los datos en la base de datos. Para esto, primero por cada
curso se obtiene la instancia correspondiente del modelo Curso, si no existe se crea.
Luego, por cada comisión se obtiene o crea al estudiante del modelo Estudiante y al guía, coguía
e integrantes del modelo Profesor. Al igual que con los cursos, se registran en la base
de datos si estos no existían previamente. Después, se obtiene o se crea la instancia de
la memoria en el modelo Memoria. Si esta se crea, se agregan al guía, coguía
e integrantes como miembros de la comisión con su respectivo rol a través del Modelo
Miembro. Mientras que si ya existía, primero se eliminan los miembros anteriores y luego
se agregan los miembros que vienen en el JSON. Este comportamiento se implementó así,
para que se puedan corregir los miembros de la comisión a través de la API.
Habiendo terminado lo anterior, se retorna un mensaje de éxito 201 Created.
El código \ref{lst:guardado_comisiones} muestra el código del guardado de las comisiones
sin las validaciones.

\begin{lstlisting}[language=Python, escapechar=|, label={lst:guardado_comisiones}, caption={Guardado de datos en la base de datos.}]
    # Guardado de datos
    for memoria in memorias:
        try:
            estudiante, _ = Estudiante.objects.get_or_create(
                nombre=nombre_normalizado(memoria.get("estudiante").get("nombre")),
                defaults={
                    "correo": memoria.get("estudiante").get("correo"),
                },
            )
            memoria_instance, memoria_created = Memoria.objects.get_or_create(curso=curso, estudiante=estudiante, tema=memoria.get("tema"))

            if not memoria_created:
                Miembro.objects.filter(memoria=memoria_instance).delete()

            if "guia" in memoria and memoria.get("guia") is not None:
                profesor, _ = Profesor.objects.get_or_create(
                    nombre=nombre_normalizado(memoria.get("guia").get("nombre")),
                    defaults={
                        "correo": memoria.get("guia").get("correo"),
                    },
                )
                Miembro.objects.create(memoria=memoria_instance, profesor=profesor, rol=rol_guia)
            if "coguia" in memoria and memoria.get("coguia") is not None:
                profesor, _ = Profesor.objects.get_or_create(
                    nombre=nombre_normalizado(memoria.get("coguia").get("nombre")),
                    defaults={
                        "correo": memoria.get("coguia").get("correo"),
                    },
                )
                Miembro.objects.create(memoria=memoria_instance, profesor=profesor, rol=rol_coguia)
            for integrante in memoria.get("integrantes"):
                profesor, _ = Profesor.objects.get_or_create(
                    nombre=nombre_normalizado(integrante.get("nombre")),
                    defaults={
                        "correo": integrante.get("correo"),
                    },
                )
                Miembro.objects.create(memoria=memoria_instance, profesor=profesor, rol=rol_integrante)

        except ValidationError as e:
            return Response(
                {"error": f"Error al guardar la memoria de '{memoria.get('estudiante').get('nombre')}': {e}"},
                status=status.HTTP_400_BAD_REQUEST,
            )
        except Exception as e:
            return Response(
                {"error": f"Error al guardar la memoria de '{memoria.get('estudiante').get('nombre')}': {e}"},
                status=status.HTTP_500_INTERNAL_SERVER_ERROR,
            )

return Response({"success": "Comisiones importadas correctamente."}, status=status.HTTP_201_CREATED)
\end{lstlisting}

Es importante mencionar que tanto en el modelo Estudiante como el modelo Profesor, se
debe buscar por nombre, pues este se define como único. Además, al guardar una instancia
de Profesor o Estudiante, se normalizan sus nombres, es decir, que todas las
palabras del string se transforman para que comiencen con mayúscula y el resto de
caracteres estén en minúscula, como se muestra en el código \ref{lst:estudiante}. Este
comportamiento estaba definido previamente en el SSM. Debido a lo anterior, al realizar
la búsqueda por nombre, se debe buscar con el nombre normalizado, ya que de lo contrario
puede que no se encuentren nombres con apellidos como ``De la Fuente'' o nombres como
``Juan-Bastián'', pues al normalizarlos se convierten en ``De La Fuente'' (la pasa a mayúscula)
y ``Juan-bastián'' (Bastián pasa a minúscula) respectivamente.

\begin{lstlisting}[language=Python, escapechar=|, label={lst:estudiante}, caption={Modelo Estudiante en el SSM.}]
  class Estudiante(models.Model):
    nombre = models.CharField(max_length=300, unique=True)
    correo = models.EmailField(max_length=200, null=True, blank=True)

    def save(self, *args, **kwargs):
      self.nombre = " ".join(word.capitalize() for word in self.nombre.split())
      super().save(*args, **kwargs)
\end{lstlisting}


Además de la implementación del procesamiento del JSON, se restringió el acceso a la
API, para que no cualquier persona que haga un POST pueda enviar datos al SSM. Con este
objetivo, se agregó autenticación con tokens al endpoint de la API y para ello, se agregó
la extensión de Django llamada Django Rest Framework (DRF) al proyecto, que proporciona
herramientas para crear APIs Restful y extiende las posibilidades de autenticación para
endpoints. Entonces, se creó un usuario para el sistema de titulación, de nombre
\verb|sistema_titulacion|, al cual se le generó un token de autenticación. Con este token
el sistema de titulación podrá autenticarse y hacer peticiones POST al endpoint de la
API. Para la autenticación es necesario enviar en el header \texttt{Authorization} de la
petición el token con el prefijo ``Token ''. Cómo se puede ver en el código
\ref{lst:decorators}, para proteger el endpoint como se mencionó anteriormente, se
utilizaron decoradores de DRF sobre la view de Django que define el endpoint. El
decorador \verb|@permission_classes| es para restringir el acceso al endpoint y el
decorador \verb|@authentication_classes| se utiliza para especificar el tipo de
autenticación que se utilizará.

\begin{lstlisting}[language=Python, escapechar=|, label={lst:decorators}, caption={Decoradores utilizados en la API.}]
@api_view(["POST"])
@permission_classes([IsAuthenticated])
@authentication_classes([TokenAuthentication])
def import_comisiones(request):
    data = request.data
    if not data or data.get("cursos") is None:
      return Response({"error": "No se encontraron datos para importar."}, status=status.HTTP_400_BAD_REQUEST)
\end{lstlisting}

Es relevante señalar que el modelo Curso del SSM incluye dentro de sus atributos la fecha
límite para la entrega del informe final de Trabajo de Título \verb|fecha_entrega_informe_final|
y cuando se crea un curso mediante la importación de comisiones por la API, este campo se
inicializa en \verb|None|, dado que dicha información no la maneja el sistema de titulación.
El modelo permite crear cursos sin fecha de entrega y es posible asignarle una fecha a cada
curso en el SSM más tarde a través de la interfaz gráfica. Sin embargo, la aplicación no
estaba diseñada para manejar cursos sin fecha de entrega, ya que varías veces se intentaba
calcular una diferencia entre fechas, causando que el sistema arrojara errores. Entonces,
se adaptó el SSM para que los casos en los que la fecha de entrega sea \verb|None| no cause
errores y en la interfaz aparezca \verb|SIN DEFINIR| en rojo y negrita que sea notorio que
se requiere asignar una fecha de entrega, como se puede ver en la figura
\ref{subfig:listado_memorias_ssm}. Además, se aprovechó el sistema de notificaciones del SSM
para enviar una notificación al usuario por cada curso sin fecha de entrega definida,
como se puede ver en la figura \ref{subfig:notificacion_ssm}.

\begin{figure}[ht]
  \centering
  \begin{subfigure}[b]{0.9\textwidth}
    \centering
    \includegraphics[width=\textwidth]{imagenes/implementaciones/ssm/listado_memorias_ssm.png}
    \caption{Comisiones de un curso sin fecha de entrega definida}
    \label{subfig:listado_memorias_ssm}
  \end{subfigure}
  \hfill
  \begin{subfigure}[b]{0.45\textwidth}
    \centering
    \includegraphics[width=\textwidth]{imagenes/implementaciones/ssm/notificacion_ssm.png}
    \caption{Notificación de fecha de entrega}
    \label{subfig:notificacion_ssm}
  \end{subfigure}
  \caption{Cambios en la interfaz del SSM para manejar cursos sin fecha de entrega}
\end{figure}


\subsubsection{Exportación de Comisiones desde el Sistema de Titulación}
La exportación de comisiones desde el Sistema de Titulación ejecuta al presionar el botón
\verb|Exportar| en la vista del listado de comisiones. Este botón realiza una llamada al
endpoint \verb|/comisiones/exportar/<periodo_id>|, donde \verb|<periodo_id>| es el id del periodo
académico que se desea exportar. El controlador de este endpoint se puede apreciar en el código
\ref{lst:export_comisiones_view}, y se encarga de llamar al comando \verb|export_comisiones_ssm|,
que tiene la tarea de realizar la exportación de comisiones, y luego redirigir al usuario al
listado de comisiones. Si el comando arroja un error, se envía un mensaje de error al usuario.
Por otro lado, si todo sale bien, se envía un mensaje de éxito al usuario. Cabe destacar que es
necesario redirigir al usuario para que el mensaje de éxito o error se muestre en la interfaz
gráfica, puesto que así está diseñado el sistema de mensajes en el sistema de titulación.

\begin{lstlisting}[language=Python, escapechar=|, label={lst:export_comisiones_view}, caption={Código del controlador que llama al comando de exportación de comisiones.}]
def get(self, request, *args, **kwargs):
  periodo = self.get_periodo()

  try:
    resultado_comando = call_command("export_comisiones_ssm", "--periodo_id", str(periodo.id))
    if resultado_comando == settings.FALLO_EN_COMANDO:
      self.msg_error("Se ha producido un error al exportar la lista de comisiones.")
      return HttpResponseRedirect(reverse_lazy("comisiones:comisiones_list", kwargs={"periodo_id": periodo.id}))
    else:
      self.msg_success("Listado de comisiones exportado correctamente.")
      return HttpResponseRedirect(reverse_lazy("comisiones:comisiones_list", kwargs={"periodo_id": periodo.id}))
  except CommandError as e:
    self.msg_error(str(e))
    return HttpResponseRedirect(reverse_lazy("comisiones:comisiones_list", kwargs={"periodo_id": periodo.id}))
  except Exception:
    self.msg_error("Se ha producido un error al exportar la lista de comisiones.")
    return HttpResponseRedirect(reverse_lazy("comisiones:comisiones_list", kwargs={"periodo_id": periodo.id}))
\end{lstlisting}

El comando \verb|export_comisiones_ssm|, que se puede apreciar en el código
\ref{lst:export_comisiones}, recibe como parámetro el periodo académico de las comisiones
que se debe exportar y en caso de no recibir ninguno, se toma el periodo académico
activo. Luego, se envían las memorias y comisiones de los estudiantes separadas por curso
en un JSON a la API del SSM. El formato del JSON es el indicado en la sección
\ref{subsubsec:adaptacion_ssm}. Solo se envían las comisiones que tengan al menos dos
evaluadores asignados. En caso de que la respuesta de la API no sea 200 ok o 201 created,
se registra el mensaje de error en el log y se arroja un \verb|CommandError| con el mensaje
de error. Por último, para incorporar las credenciales de la API del SSM, se agregaron
las variables de entorno \verb|SSM_URL| y \verb|SSM_TOKEN| que corresponden a la URL y el
token de la API del SSM, respectivamente.

\newpage
\begin{lstlisting}[language=Python, escapechar=|, label={lst:export_comisiones}, caption={Comando de exportación de comisiones.}]
def handle(self, *args, **options):
  periodo_id = options.get("periodo_id")
  if periodo_id is None:
    periodo = Periodo.objects.filter(is_activo=True).first()
  else:
    try:
      periodo = Periodo.objects.get(id=periodo_id)
    except Periodo.DoesNotExist:
      self.stdout.write(self.style.ERROR("Periodo no encontrado."))
      return settings.FALLO_EN_COMANDO

  cursos = Curso.objects.select_related("periodo").filter(periodo=periodo)
  memorias = MemoriaEnF.objects.select_related(
    "estudiante__persona",
    "tema__guia__persona",
    "tema__coguia__persona"
  ).prefetch_related(
    "comision__evaluadores__persona"
  ).filter(periodo=periodo)

  comisiones_por_curso = {
    "cursos": [
      {
        "codigo": curso.codigo,
        "seccion": curso.seccion,
        "semestre": "O" if curso.periodo.periodo == 1 else "P",
        "anno": curso.periodo.ano,
        "memorias": [
          memoria.to_ssm_dict() for memoria in memorias.filter(curso=curso) if memoria.comision.evaluadores.count() >= 2
        ]
      } for curso in cursos
    ]

  }

  response = requests.post(
    settings.SSM_URL + "/api/comisiones/",
    json=comisiones_por_curso,
    headers={"Authorization": f"Token {settings.SSM_TOKEN}"},
  )

  if response.status_code not in (requests.codes.ok, requests.codes.created):
    self.stdout.write(self.style.ERROR(f"Error al exportar comisiones: {response.text}"))
    raise CommandError(response.json()["error"])

  # Establece la |comisi\'on| de un estudiante como publicada y se podr\'a ver en la ficha del estudiante
  for memoria in memorias:
    memoria.comision.publicada = True
    memoria.comision.save()

  self.stdout.write(self.style.SUCCESS("Comisiones exportadas correctamente"))

  return 0
\end{lstlisting}

\chapter{Validación}\label{cap:validacion}
\section{Evaluación}
\begin{itemize}
    \item Objetivo: Asignar comisiones examinadoras de estudiantes de Trabajo de Título
    \item Objetivo: Exportar las comisiones asignadas a un archivo CSV
    \item Asignación de comisiones examinadoras de estudiantes de Trabajo de Título Primavera 2025.
    \item Generar CSV con comisiones asignadas.
\end{itemize}
\section{Retroalimentación}
\begin{itemize}
    \item Comentarios de la coordinadora de titulación
    \item Correcciones descritas en implementación
\end{itemize}


\chapter{Conclusiones}
El desarrollo del módulo de asignación de comisiones examinadoras en el sistema de
titulación del DCC tuvo como objetivo general desarrollar y desplegar una herramienta que
permita asignar comisiones examinadoras a los estudiantes de Trabajo de Título de forma
eficiente y que se comunique con los sistemas existentes relacionados con el proceso de
titulación. Este objetivo fue logrado con éxito, ya que se implementó una extensión del
sistema de titulación del DCC que permite a la coordinadora de titulación asignar
comisiones examinadoras a los estudiantes de Trabajo de Título en menor tiempo que el
método anterior de planillas Excel y que es capaz de exportar las comisiones asignadas
al Sistema de Seguimiento de Memorias por dos métodos: la descarga de un archivo CSV y
mediante una consulta POST a la API del Sistema de Seguimiento de Memorias. Además, se
desplegó la herramienta en el servidor de producción del DCC, dejándola lista para su uso.

Dentro de los aprendizajes obtenidos, se destaca la importancia de tener reuniones periódicas
con la profesora guía para mantener la constancia en el trabajo, ya que inicialmente no
se tenían y en cuanto se fijó un horario para tener reuniones semanales, el ritmo de trabajo
aumentó. Las reuniones, tanto con la profesora guía como con el equipo de desarrollo del DCC,
requirieron de saber expresar lo que se estaba haciendo y los problemas que surgían, para así
obtener ayuda o sugerencias que permitieran resolver las dificultades que se tuvieron.

Uno de los principales desafíos de la memoria fue trabajar sobre código desarrollado por
otras personas, en particular trabajar sobre el código del piloto de asignación de
comisiones examinadoras, ya que no se tenía una documentación detallada del código y al
ser una versión piloto, funcionaba pero tenía errores que se debían corregir y
optimizar. Gran parte del tiempo dedicado al desarrollo se centró en corregir errores que
surgían de casos borde no considerados en el piloto u optimizar el código para que fuera
más eficiente.

En cuanto a aspectos a mejorar y trabajos futuros en la asignación de comisiones examinadoras,
primero se deben realizar mejoras en la usabilidad de la herramienta, por ejemplo,
mientras se obtienen los estudiantes de Trabajo de Título o mientras se exportan las
comisiones al SSM no se muestra ningún indicador que permita al usuario saber que la
operación está en curso. Otro aspecto a mejorar son dos comportamientos no deseados
presentes en el sistema que no hacen que la herramienta se caiga. El primero aparece cuando
se obtienen los estudiantes de Trabajo de Título con el botón Sincronizar en el listado de
comisiones. Al completar la operación, se redirige al usuario al listado de estudiantes de
Trabajo de Título en vez de simplemente actualizar la página. El segundo aparece al
guardar una comisión examinadora y se tiene algún error de validación. En este caso puede
que la alerta de error se muestre más de una vez. Un último trabajo futuro sería incorporar
el procesamiento de solicitudes de vía rápida a través del sistema de titulación. Esto afecta
al módulo de Trabajo de Título, ya que se deben haber ingresado al sistema de titulación los
temas de los estudiantes que desean solicitar vía rápida, para así poder identificarlos correctamente
como estudiantes de Trabajo de Título y poder asignarles comisiones examinadoras.

A modo de resumen, el desarrollo del módulo de asignación de comisiones examinadoras en el
sistema de titulación del DCC tuvo éxito en cumplir su objetivo general. Este trabajo aporta
a que el Departamento de Ciencias de la Computación tenga herramientas más adecuadas
ante el aumento de estudiantes que ingresan a la carrera de Ingeniería Civil en Computación,
como lo son el sistema de titulación y el sistema de seguimiento de memorias.
Además, permite que el sistema de titulación del DCC no solo cubra el ramo de Introducción al
Trabajo de Título, sino que también cubra parte del ramo Trabajo de Título. Esto abre puertas para
agregar más funcionalidades al sistema de titulación relacionadas con Trabajo de Título.

% ver https://www.overleaf.com/learn/latex/Glossaries
% \input{glosario.tex} % opcional

\nocite{*}
\bibliographystyle{plain}
\bibliography{bibliografia}

% opcional ...
% \begin{appendices}
%     \input{secciones/anexoA.tex}
% \end{appendices}
\end{document}
